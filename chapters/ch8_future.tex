\chapter[Future Directions]{Future Directions} \label{ch:future}

Here we discuss the possible future directions of the work presented in this dissertation.

\section{Massive Black Hole Binaries (MBHBs)} 

Part I of this thesis models the interaction of a gas disc with a binary in
order to determine electromagnetic signatures of accretion onto MBHBs.
Chapters \ref{ch:CavAcc} and \ref{ch:CBDTrans} do this by numerically solving
the equations of isothermal hydrodynamics in the presence of two point masses
in a circular, Newtonian orbit. As discussed in Chapter
\ref{ch:intro}, a number of other works have considered the addition of more
sophisticated physics in circumbinary disc simulations. These include
extension to three dimensions, inclusion of magnetic fields (magneto-hydrodynamics), 
non-isothermal energy prescriptions, the back-reaction of the binary over short
time periods, and General Relativity (see Chapter \ref{ch:intro}). This
incremental inclusion of complexity should continue to march forward and to
deepen our understanding of binary accretion. However, each of these studies
use idealized initial conditions to simulate the response of the gas to a
binary on a fixed orbit, or allow the binary and gas to feedback on each
other, but only for timescales which are short compared to the time needed for
the binary to ``migrate'' through the disc
\citep[\textit{e.g.},][]{RoedigSGmigrate:2014}.




To make robust predictions of the electromagnetic signatures of a MBHB+disc, and also to
predict merger rates influenced by interactions with gas, the
long-term evolution of the coupled binary+disc system must be taken into
account.  However, a fully coupled, long term evolution of the binary+disc has
only been realized in semi-analytic, 1D solutions which assume the gas reaches
a steady-state at each point in the binary's orbital evolution, and that local
thermodynamic equilibrium is always achieved \citep{KHL:2012:MNRAS_PI}.
Even with these restrictions, important new phenomena emerge when the binary
orbit and disc structure are coupled over the macroscopic timescales needed
for of-order-unity changes in orbital separation \citep{KHL:2012:MNRAS_PII}.
Future work is planned to create the first multi-dimensional simulations 
that crucially track this mutual evolution of the binary+disc. 

Methodically relaxing approximations to models of the MBHB+disc system in a
series of one-, two-, and three-dimensional simulations will teach us about
the coupled evolution of the system as it draws towards merger and inform us
of the correct initial conditions and their importance for the above
idealized simulations.

% that start with a close binary and relax the disc to quasi-equilibrium over short timescales.






Presently work is being completed to apply the models for reverberated infrared (IR)
emission from periodic MBHB sources (Chapter \ref{ch:Dust}) to the MBHB
candidate PG 1302. These models will test whether the IR emission from PG 1302
is consistent with reverberation by a periodic UV/optical central source. If
so, these models can determine whether the central source is varying
isotropically, as in the case of a time variable accretion rate (Chapters
\ref{ch:CavAcc} through \ref{ch:PG1302_a}) or anisotropically, as it would in
the Doppler-boost scenario (Chapter \ref{ch:PG1302_b}), and also constrain the
geometry and make-up of the surrounding dust. After application to PG1302, we
can vet the remaining $\sim100$ MBHB candidates \citep{Graham+2015b,
Charisi+2016} with our IR reverberation models. Work should also be done to
scour IR surveys for periodic IR variability in the absence of ultra-violet/optical
periodicity as predicted in Chapter \ref{ch:Dust}.


Future work is planned to extend the models of Chapter \ref{ch:Dust}, for
IR dust reverberation, to the related case of reverberation by the broad-line
region of active galactic nuclei. Such a study could yield unique predictions
for the morphology and dynamics of broad lines from regions surrounding MBHBs.

Additional work is planned to determine the importance of general relativity
on the observed light curves of accreting MBHBs. For example, time delays,
lensing, and orbital precession may all affect UV/optical as well as
reverberated IR light curves, and each waveband could be affected differently.
Such work could yield new ways to seek out MBHBs and distinguish them from
single MBHs, and it may may hold consequences for the inferences made about
the population of MBHBs.

%DD lensing affect dust reverberation! inclination doesn't matter, dust shell will always see some dust lensing?


\section{Black Hole Neutron Star (BHNS) Binaries }

The second part of this dissertation concerned modeling BHNS magnetospheres
with the aim to predict electromagnetic emission from their mergers.
The main road block in developing more robust models for BHNS magnetospheric
emission is the problem of dissipation: where and how is the available black-hole-battery 
energy dissipated. This is a difficult question to address because it
involves, at the most basic level, understanding the dynamics of strong
electromagnetic fields and an electron-positron pair plasma in a binary
spacetime. A similar problem is still at the heart of a complete theoretical
description of a the magnetosphere of a single spinning neutron star, where a break down
of force-free electrodynamics in regions of the magnetosphere called `gaps'
allow particle acceleration. Depending on the positioning of the gap in the
magnetosphere, pulsar emission can take on different characteristics
\citep[\textit{e.g.},][and references therein]{YukiGaps:2012}. Without observations to
yet guide work on the BHNS magnetosphere, the task of modeling dissipation
mechanisms becomes even more difficult, but the possibilities should be
explored

An analytic investigation of the basic properties of a magnetosphere with one
event horizon and a light cylinder due to orbital motion may yield global
information on the magnetosphere,  Such an investigation could lend insight
into where dissipation may occur in the BHNS magnetosphere, much as early work
on pulsar magnetospheres suggested the existence of a polar gap
\citep{RudSuth:1975}. Further knowledge on where dissipation may occur could
be gained from force-free simulations of the BHNS magnetosphere  \citep[one
such simulation has already been carried out by][]{Paschalidis:2013}.  Full
information on particle acceleration however requires the breakdown of force-free 
electrodynamics and will rely on particle-in-cell  simulations which
track the trajectories of electron-positron pairs under the influence of
evolving electric and magnetic fields. Such particle-in-cell simulations must
also implement a prescription for pair production which is crucial for
understanding screening of electromagnetic fields and the eventual ignition of
a fireball. The future may also look to magnetic reconnection or more
complicated methods to dissipate the black-hole-battery power into observable
electromagnetic radiation.


%
%DD: think theorems for force-free electrodynamics trapped by two light surfaces. 
%

If we are willing to except that a near-unity fraction of the black-hole-battery 
power is injected into powering fireballs, then the models of Chapter
\ref{ch:NSBH_Fireball} could be built upon further.  In Chapter
\ref{ch:NSBH_Fireball} we predicted the form of electromagnetic emission from
a disc merger by following the simplest model of a pure radiation fireball,
first set forth in the context of gamma-ray bursts (GRBs) by \cite{Pacz:1986GRB} and
\cite{GoodmanGRB:1986}. In this scenario, energy injection creates a
pair+photon fluid that is optically thick to pair production causing
relativistic expansion until the fluid becomes transparent and photons escape;
a thermal spectrum results. Observations of non-thermal emission from GRBs
pushed theorists to develop fireball models in which baryonic
matter may accompany the pair+photon fluid resulting in shocks between the
expanding fireball and itself or an external medium. These shocks result in
non-thermal emission, which comes closer to that observed in GRBs. Future work
should employ the theoretical framework describing baryonic fireballs
\citep{Piran_FBRev:1999} to the disc system in order to better understand
possible emission from disc fireballs.


In conclusion, this dissertation has been completed at an exciting time for
astrophysics, at the inception of the era of multi-messenger astronomy. In the
near future the observation of not only electromagnetic, but also
gravitational radiation, as well as high energy neutrinos and cosmic rays,
will provide the most complete view we have yet had of the universe. It is my
hope that the systems discussed here, and the work presented in this thesis,
will contribute to this endeavor.

