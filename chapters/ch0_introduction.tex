\chapter[Introduction]{Introduction} \label{ch:intro}

%?There are those who love to get dirty and fix things. They drink coffee at dawn, beer after work. And those who stay clean, just appreciate things. At breakfast they have milk and juice at night. There are those who do both, they drink tea.? 
%? Gary Snyder

%?Nature is not a place to visit. It is home.? 
%? Gary Snyder


%?Three-fourths of philosophy and literature is the talk of people trying to convince themselves that they really like the cage they were tricked into entering.? 
%? Gary Snyder


%?With no surroundings there can be no path, and with no path one cannot become free.? 
%? Gary Snyder, Practice of the Wild

\vspace{-16pt} \begin{chapquote}{Gary Snyder quoting Ezra Pound para-phrasing Lu Ji} \singlespacing ``When making an axe handle, the pattern is near at hand.'' 
 \end{chapquote} \vspace{-8pt}
\noindent\makebox[\linewidth]{\rule{0.5\textwidth}{0.5pt}} \vspace{1pt}

At 09:50 UTC on September 14, 2015, the laser interferometer gravitational wave observatory (LIGO) detected gravitational waves from the merger of two $\sim 30 \msun$ black holes. This first direct detection of gravitational radiation (gravitational waves or GWs for short) and first unequivocal confirmation of the existence of black holes has ushered in a new era of gravitational wave astronomy. In this era, multi-wavelength and multi-messenger (photons, gravitons, neutrinos) observations of GW sources will be crucial for making the next steps to...

Human knowledge of the ... has come largely from our ability to sense photons.... into GWs

Gravitational radiation is generated when the shape of a mass-energy distribution changes in time. More precisely, when quadrapole and higher moments of a mass distribution have a changing current, metric perturbations
\begin{equation}
h_{i j} = \frac{4G}{d c^4} \ddot{Q_{ij}},
\end{equation}
are generated and propagates through space as gravitational waves. Here $G$ and $c$ are the usual gravitational constant and the speed of light, while $d$ is the distance from observer to source of radiation, and $Q_{ij}$ is quadrapole tensor. The units of $\ddot{Q_{ij}}$ are a mass times a velocity squared. Hence the wave amplitude can be thought of as twice the kinetic energy $M v^2$ put into changing the quadrapolar moment of a mass-energy distribution times a coupling constant $4 Gc^{-4}d^{-1}$. The minuscule size of this coupling constant is perhaps the reason that it has taken a century since their prediction to detect gravitational waves. That is, to experience gravitational wave amplitudes of order unity, one must put a detector at a distance $d=GMc^{-2} (c/v)^2$ from a system of mass $M$ and with typical velocity $v$ contributing to qudrapolar change. This is distance is of order the gravitational radius, telling us that the largest possible gravitational waves are generated by mass distributions with velocities approaching the speed of light and packed into a space the size of a black hole event horizon. GWs with larger amplitudes are shielded by an event horizon. Hence, in building a gravitational wave emitter, black holes are the golden standard in components. As we have not yet built black holes in a laboratory we look to astrophysical sources. The best known astrophysical sources which approach the dimensions discussed above are
\begin{itemize}
\item Mergers of two (or more) compact objects, namely black holes, neutron stars and white dwarfs \citep{}. At the time of writing this class of sources is the only to have been detected in gravitational waves\citep{GW091415}. Two such binary systems are the subjects of this thesis.
\item Inflation \citep[\textit{e.g.}][]{Guzzatti:2016}.
\item Cosmological defects such as comic strings  \citep[\textit{e.g.}][]{}.
\item Neutron star mountains  \citep[\textit{e.g.}][]{}.
\item Supernovae  \citep[\textit{e.g.}][]{}.
\end{itemize}


The methods for detecting gravitational waves is nearly as varied as the sources themselves. Just as for EM radiation detectors, the detector design depends on the radiation frequency. For the gravitational wave source most relevant to this these, coalescing binary systems, the gravitational wave frequency is given by,
\begin{equation}
\end{equation}
Section on GW detectors and prospects
LIGO and Einstein
eLISA
PTAs
others?





by a non-vanishing second derivative of the energy density quadruple moment in time. The amplitude of gravitational waves (GWs) which compose this radiation is dependent on the mass and rate at which this mass can be moved to change the system's quadruple moment. 


To get a feel for the types of systems which can generate observable gravitational waves, note that the dimensionless gravitational wave strain is proportional to the second time derivative of the quadrapole moment by a coupling constant of order $2Gc^{-4}d^{-1} \sim 10^{10^{-50}}$ or \sim 10^{-76} d_{\rm{Mpc}}$g^{-1}(cm/s)^{-2}, where $G$ is the gravitational constant, c is the speed of light and $d$ is the distance to the source of GWs, calculated for a laboratory sized distance of one centimeter $d_{\rm{cm}}$ and an astrophysical distance of 100 Mpc $d_{\rm{Mpc}}$. The second time derivative of the quadrapole moment is proportional to the mass of the GW source times the square of the typical system velocity. Then for a laboratory source with a mass of 1 kg that we accelerate to near the speed of light, a gravitational wave strain of only $10^{-25}$ could be measured at a distance of 1 cm from the source. This is an impausibale if not impossible task.%It is of course implausible to accelerate 1Kg up to near the speed of light and measure a distance change on this size scale. 
Instead, imagine an astrophysical source which can be very massive and moving near the speed of light but at the cost of being a further distance from our detectors. In the astrophysical case, an object at 100 Mpc, consisting of 10 solar masses moving at near the speed of light, still only generates a waves strain of $10^{-20}$, but this is enough to be detected by LIGO over a frequency range of Hz to KHZ. Astrophysical sources which can generate this level of strain are
%In gravity the monopole is mass which has a 0 time derivative (or does it?), the time derivative of the dipole moment is momentum and so must be 0
Section on sources of gravitational waves
\begin{itemize}
\item Mergers of two (or more) compact objects, namely black holes, neutron stars and white dwarfs. - the subjects of this thesis fall into this category.
\item Inflation
\item Cosmological defects such as comic strings.
\item Neutron star mountains.
\item Supernovae
\end{itemize}

The broad motivation of this thesis is to examine astrophysical sources of gravitational radiation in the context of their astrophysical environments and, in this way, deduce the possibilities for EM observations of GW events which could produce multi-messenger events or serve as pure EM observations identifiers of GW events in absence of a GW detection.

But what can we do with EM detections of gravitational wave sources? A goal of this thesis is indeed to elucidate this question for two specific cases of neutron star black hole (NSBH) binaries and massive black hole binaries (MBHBs). In order to place this goal in greater context, we first survey the known literature on uses of EM signatures of GW sources. It is useful to separate these into two categories

\paragraph{EM counterparts} When an EM signal is observable in close temporal proximity to an observable GW event, we call this an EM counterpart because it is a counterpart to the GWs (I suppose we could just as likely call this a GW counterpart to an EM source). Examples in the literature include:
\begin{itemize}
\item gas squeezing X-ray flare at merger (Menou Cheng, snowplough papers?)
\item Disk response to BH recoil
\item SGRBs
\item ...
\end{itemize}
as well as the BH battery mechanism detailed in \ref{} of this thesis.

\paragraph{EM only} In the case that EM radiation can identify the a GW source 

 



%This thesis has focused on predicting the observable electromagnetic signatures of two specific sources of gravitational radiation.  So I need to :


1)  Introduce the concept of gravitational radiation and its importance.
	a) what are GWs
	b) What are the expected sources of GWs
	c) How are GWs detected - list of experiments and their sensitivity (Figure?)
2) Introduce the role EM counterparts can play 
	a) with GW detection
	b) without GW detection
	
The specific sources of GWs studied in this thesis are the inspiral and merger of MBHBs in galactic nuclei and the merger of magnetized NSs with $\gsim 10 \Msun$ BHs.
\section{Part I: Massive Black Hole Binaries}
Origin story + Final Parsec Problem

List of past work:
Theory of disk+binary interaction
Simulations

The first observations- Graham, D'Orazio, Charisi



\section{Part II: Stellar Black Hole + Neutron Star Binaries.}

Tidal Disruption details - and EM signals from disruption

Explanation of BH battery - Review of UI models in astro

review of GJ69 and Force Free?

Review of Pop Synth for NSBH LIGO


prospects in the LIGO era









\section{Outline of thesis}























