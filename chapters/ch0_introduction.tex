\chapter[Overview]{Overview} \label{ch:intro}
%
%labels: 
%CH1: ch:CavAcc
%CH2: ch:CBDTrans
%ch:PG1302_a
%ch:PG1302_b
%ch:Dust
%ch:Rindler
%ch:NSBH_Fireball
%
%?There are those who love to get dirty and fix things. They drink coffee at dawn, beer after work. And those who stay clean, just appreciate things. At breakfast they have milk and juice at night. There are those who do both, they drink tea.? 
%? Gary Snyder
%
%?Nature is not a place to visit. It is home.? 
%? Gary Snyder
%
%
%?Three-fourths of philosophy and literature is the talk of people trying to convince themselves that they really like the cage they were tricked into entering.? 
%? Gary Snyder
%
%
%?With no surroundings there can be no path, and with no path one cannot become free.? 
%? Gary Snyder, Practice of the Wild
%
\vspace{-16pt} \begin{chapquote}{Lu Ji} \singlespacing ``When making an axe handle, the pattern is near at hand.'' 
 \end{chapquote} \vspace{-8pt}
\noindent\makebox[\linewidth]{\rule{0.5\textwidth}{0.5pt}} \vspace{1pt}

Scientific discovery is driven by observations. The majority of these
observations and corresponding scientific conclusions are founded on the
detection of photons, the messenger of the electromagnetic (EM) interaction.
Recently, our electromagnetic view of the universe has been supplemented by
multi-messenger observations of neutrinos, cosmic rays, and recently,
gravitational radiation.

%and only within one year of writing, gravitational radiation.

Over the last $\sim 60$ years, neutrino astronomy has taught us a great deal
about fundamental particle physics, the power source of the sun, and the death
of massive stars. The first non-terrestrial neutrinos were observed from the
Sun and were instrumental in the discovery of neutrino flavor oscillations and
establishing our understanding of the Sun's central power source
\citep{Haxton:SolarNeutrinos:2013}. The first extra-solar neutrinos were
observed from supernova 1987A, serving as a precursor to the explosion in the
optical. This multi-messenger view of the death of a massive star put further
limits on the fundamental properties of the neutrino and enhanced our
understanding of supernovae \citep{Hirata:1987, Bionta:1987}.




Present and future neutrino observatories that aim to observe ultra-high
energy neutrinos generated from cosmic sources promise to open a window into
the engines which power the brightest objects in the universe: \textit{e.g.},
gamma-ray bursts (GRBs) and active galactic nuclei (AGN). The first detection
of a high-energy ($\gsim10^6 \times$ the energy of neutrinos from 1987A) flux
of astrophysical neutrinos was made by ICECUBE over a period from
2010-2013 \citep{ICECUBE:2013:detection}. Future neutrino experiments may be
able to isolate sources of high-energy astrophysical neutrinos and associate
them with optical or perhaps even gravitational wave counterparts, teaching us
about their production and the environments which produce them. In the future,
neutrino observations may also be key in unraveling the mystery of dark matter
and could serve as cosmological messenger beyond the surface of last photon
scattering \citep{HEneutrino_Rev:2011}.










% Studying neutrinos can teach us about Dark Matter, Cosmic rays, fundamental particle physics, and also the astrophysical objects which generate them:

% Cosmic neutrinos from the Sun were observed for the first time in 1968 in the Homestake experiment teaching us

% Bahcall predicted that the solar neutrino flux should be tree times higher than detected in Davis's experiment. This lead to the discovery of neutrino oscillations between three flavors of which DAvis's experiment could only detect one 
% flavor mixing and solar power source. Other experiments that look at the solar neutrino flux and added to this discovery were the Kamiokonde experiment, the super-Kamiokande experiment, and the Sudbury neutrino observatory.

% atmospheric neutrinos

% The first extra-solar neutrinos were observed from supernova 1987A and taught us...

% presently the ICECBE detector in the south pole has collected the first ...
% first solid evidence for astrophysical neutrinos from cosmic accelerators
% $>10^6$ 1987A energies
% first astrophysical high-energy neutrino flux ever

% likely generated in gamma-ray bursts, or the engines which power active galactic nuclei.

% Future upgrades will allow us to pin down the sources of high energy neutrino emission teaching more about the physical processes which drive the highest energy events in the universe: GRBs, AGN, supernovae, and 


% Neutrino paragraph:
% Past
% Solar \citep{Haxton:SolarNeutrinos:2013}
% Nobel prize
% 1987 A neutrinos \citep{Hirata:1987, Bionta:1987}

% Present
% IceCube - \citep{ICECUBE:2013:detection}

% Future
% low Energy CMB neutrinos
% PINGU (ICECUBE ext.)
% expand underwater neutrino observatory in the Mediterranean called Antares. 
% a large-scale observatory in a lake in Siberia.
% eventually establish the Askaryan Radio Array, a 100-cubic-kilometer neutrino detector in Antarctica
% NESTOR+Nemo->KM3NeT framework in mediteranean
% Both KM3NeT and GVD could be completed by 2017 and it is expected that all three will form a global neutrino observatory


% Sources:
% AGN
% SN
% accreting NSs
% CMB

%Solar neutrinos observations turned out to be hugely important for fundamental physics in detecting evidence for flavor mixing and also helped establish a basic understanding of the nuclear physics providing the power source at the center of the Sun. Neutrino observations of 1987A established the formation and sustained life of a proto neutron star following core collapse of a massive star, thus confirming the basic picture of massive stellar death and explosion. Current high-energy neutrino observations constrain important physical processes including GZK. SuperK and others are well posed to observe the next nearby core collapse SN and perhaps stochastic backgrounds from the integrated universe.% 
% Especially important is that combined neutrino/GW/EM (if possible) are key to fully understanding GW sources in the future.
%
% I strongly suggest that you research and write a paragraph properly acknowledging the role of neutrino astrophysics in the past, present and future.  



%Before 2015, all such
%observations and corresponding scientific conclusions were founded on the
%detection of photons, the messenger of the electromagnetic (EM)
%interaction.\footnote{
%plus a few neutrinos
%\citep{Haxton:SolarNeutrinos:2013, Hirata:1987, Bionta:1987, ICECUBE:2013:detection}}



Only within the last year of writing has the newest cosmic messenger been
added to our list of tools with which to study the universe. At 09:50 UTC on
September 14, 2015 the laser interferometer gravitational wave observatory
(LIGO) observed the universe for the first time in gravitons, or gravity waves
(GWs), the messenger of the gravitational interaction \citep{GW150914:2016}.
Even before the detection of GWs, the importance of combining information from
EM, neutrino, and gravitational views was recognized
\citep[\textit{e.g.}][]{ThorneBraginsky:1976,Phinney:2009, GW+Neutrino:2010}.
The work laid out in this thesis is a contribution to this effort, to maximize
science returns from observations of GW sources by predicting the nature of EM
signatures that should accompany them, or signify their existence beforehand.
Such an endeavor not only provides ways to find sources of GWs and learn about
their operation, but also drives investigation into the astrophysics that
creates GW sources, and into the workings of physical processes in the extreme
environments that generate gravitational radiation. We proceed by briefly
discussing the expected sources of GWs, their detection, and the utility of
their possible EM signatures. We then introduce two specific GW sources that
are the topic of this thesis.



\subsection{The gravitational side} 

Gravitational radiation is generated by non-zero second time derivatives of quadrupole or
higher moments of a mass-energy distribution. The result is the generation of
metric perturbations
\begin{equation}  
h_{i j} = \frac{2G}{d c^4}\ddot{Q}_{ij},  
\end{equation}
that propagate through spacetime as gravitational waves, carrying the
information of a changing gravitational field
\citep[\textit{e.g.}]{WaldGR:1984}. Here $G$ and $c$ are the usual gravitational
constant and the speed of light, while $d$ is the distance from observer to
source of radiation, $\ddot{Q}_{ij}$ is the second time derivative of the
mass-energy quadrupole tensor and $h_{ij}$ is the dimensionless strain tensor
which measures fractional changes in proper distances. The units of
$\ddot{Q}_{ij}$ are a mass times a velocity squared. Hence you can envision
the wave amplitude as being set by the kinetic energy put into
creating a doubly time changing quadrupolar moment of a mass-energy distribution, times a
coupling constant $4 Gc^{-4}d^{-1}$. The coupling constant is determined by
the strength of the quadrupolar tidal field; the minuscule size of this
coupling constant is perhaps the reason why it has taken a century since their
prediction to detect gravitational waves. For example, the gravitational wave
strain from two point masses of total mass M, on a circular orbit of
separation $a$ is of order \begin{equation}  h \sim \frac{2G}{d c^4} M v^2
\sim \frac{G M }{a c^2} \frac{G M }{d c^2}.  \label{Eq:BinStrain}
\end{equation} Even for a binary consisting of two Suns, orbiting as rapidly
as possible, $a=R_{\odot}$, and within our galaxy $d \sim 1$ kpc the strain is
incredibly small: $h\sim10^{-22}$.



To experience gravitational wave strains of order unity, one must put a
detector at a distance $d= 2GMc^{-2} (v/c)^2$ from a system of mass $M$ and
with typical velocities $v$; of order unity strains can only be experienced
near or within the event horizon of a black hole.
%
%This distance is of order the gravitational
%radius, or the event horizon scale of a black hole. 
%
As we have not yet built
black holes in a laboratory, we look to astrophysical sources. The best known
astrophysical sources that approach these dimensions, and could occur close
enough and frequently enough to be detectable, are the mergers of two (or
more) compact objects, namely black holes (BHs), neutron stars (NSs), and
white dwarfs (WDs) \citep[\textit{e.g.}][]{ThorneBraginsky:1976,
ClarkeErdley:1977, Belczynski:2016}, cosmic inflation \citep{Starobinski:1979}
or \citep[\textit{e.g.}][for a recent review]{Guzzetti:2016}, cosmological
defects such as cosmic strings \citep[\textit{e.g.}][and references
therein]{Damour:2005},  non-axisymmetric features of rapidly spinning Neutron
stars \citep[\textit{e.g.}][and references therein]{Haskell:2015}, and core-
collapse supernovae \citep[\textit{e.g.}][and references
therein]{FryerNew:2003:LRR}.  For the remainder of this thesis we focus on the
first example, and specifically  the mergers BHs and NSs binaries.

The are multiple methods for detecting gravitational waves from merging
compact objects. Just as for EM radiation, detector design depends on the
radiation frequency. The gravitational wave frequency for a binary on a
circular orbit is given by twice the orbital frequency\footnote{Eccentric
orbits emit gravitational waves over a spectrum of frequencies spanning the
circular frequency and its higher order harmonics \citep[\textit{e.g.}][]{Enoki:2007}.} 
\begin{equation}
f_{\GW} = 2 f_{\orb} \approx  \frac{1}{\pi} \sqrt{\frac{G M }{a^3}} =
\frac{1}{\pi} t^{-1}_{\rm{G}} \left( \frac{a}{r_{\rm{G}}} \right)^{-3/2}
\label{Eq:GWfreq}
\end{equation} 
where $M$ is the total binary mass and $a$ is the binary
separation and $t_{\rm{G}} \equiv GMc^{-3}$ is the gravitational time while
$r_{\rm{G}} \equiv GMc^{-2}$ is the gravitational radius.  For astrophysical
black holes, which range in mass from $\sim 1 \Msun \rightarrow 10^{10}
\Msun$, the gravitational wave frequency covers ten orders of magnitude.
Assuming $a = 2GM/c^2$ at merger, this range gives $f_{\GW} = 10^4 \rightarrow
10^{-6}$ Hz. Considering also GW emission during the inspiral stage, the
largest black holes emit at frequencies of $f_{\GW} \sim 10^{-9}$ Hz at
separations of order $100 GM/c^2$.

This wide range of astrophysically interesting frequencies is currently
covered by three different detector designs. From high to low frequencies, the
first two use laser interferometers to detect the very small distance change
between two test masses when a GW passes through them. The laser
interferometer gravitational wave observatory (LIGO) is sensitive to GW
frequencies ranging from $\sim 10 \rightarrow 10^4$ Hz with a peak strain
sensitivity at $\sim10^{2}$ Hz of $h \gsim 10^{-22}$ \citep{aLIGO:2015}. This
makes LIGO sensitive to the inspiral, merger, and ringdown of stellar mass
compact object binaries consisting of BHs and NSs. 
%
%http://arxiv.org/pdf/1211.6427v2.pdf - high freq tails MBH GWs for LIGO
%%GWs from the harmonics of eccentric MBHB inspirals, 
%
LIGO could also detect the mountains on millisecond pulsars 
\citep[\textit{e.g.}][and references therein]{ContWaveLIGO:2016}, 
or the stellar oscillations due to giant core
collapse supernovae \citep[\textit{e.g.}][and references
therein]{SNLIGO:2016}. LIGO's localization capabilities are currently limited
to a rather broad $\sim$few 100 square degrees, but could improve to $\sim$few
square degrees \citep{LIGO_Loc:2016} when the two existing interferometers are
joined by their international counterparts: VIRGO \citep{Acernese:2015} in
Italy, GEO600 in Germany \citep{Dooley:2015}, KAGRA being built in Japan
\citep{Tomaru:2016}, and in LIGO-India approved in March of 2016
\citep{LIGOIndia}.

At frequencies below $\sim 1$ Hz, earth related vibrations swamp the LIGO
sensitivity making detection of sub Hz sources impossible \citep{aLIGO:2015}.
For this reason, space-based interferometers were envisioned
\citep{ThorneBraginsky:1976}. Presently, the leading design is embodied in the
eLISA mission, planned to be sensitive to GWs with frequency in the range
$10^{-5} \rightarrow 1$ Hz with a peak sensitivity of $h \gsim 10^{-23}$, over
a range of $0.01 \rightarrow 0.1$ Hz \citep{eLISA:AmaroSeoane:2013}. eLISA
will be oriented in a orbit around the Sun such that its changing orientation
in time will allow localization of sources to $\lsim10$ square degree
\citep{eLISA:AmaroSeoane:2013, KleinLISA:2016}, or to less than a square degree for longer
observations of inspiraling stellar mass BHs \citep{{Sesana:LISALIGO:2016}}.
%
%DD: check ref on this comment form zoltan below
%or even to arcminutes using spin 
%precession and higher waveform harmonics \citep{}. 
%
LISA sources include the inspiral and merger
of $10^4 \rightarrow 10^7 \Msun/(1+z)$ MBHBs in galactic nuclei at redshift
$z$, the orbits of thousands of galactic binaries, extreme mass ratio
inspirals of compact objects, stochastic GWs from the early universe
\citep{eLISA:AmaroSeoane:2013}, and the inspiral of NS and stellar BH binaries
before they reach the LIGO band \citep[\textit{e.g.}][and references therein]{Sesana:LISALIGO:2016}.

A second type of gravitational wave detector looks to nature's clocks, the
pulsars, to act as a galactic timing array. The so-called Pulsar Timing Arrays
(PTAs) search for deviations in the arrival time of the pulses from
millisecond pulsars. Timing deviations on the order of nano-seconds,
correlated over multiple pulsars in our galaxy would signify the presence of
very long wavelength gravitational waves, with frequencies ranging from
$\sim10^{-9} \rightarrow 10^{-6}$ Hz (wavelengths of parsecs to mill-
parsecs!). Such low frequency radiation is expected from the inspiral of the
largest BHs in the universe in galactic centers. For the closest ($z\lsim1$)
MBHB inspirals, the PTAs could pick out the GW signal from an individual
event, otherwise the PTAs will measure a stochastic background of GWs from
MBHBs spiraling together throughout the universe. The magnitude and frequency
dependence of the GW background holds information on the role of gas and stars
in driving the binary inspiral through the PTA band and is an important probe
of the MBHB population \citep[\textit{e.g.}][]{Sesana:MBHB_PTA:2015}. The PTAs
may also be sensitive to more exotic sources of gravitational radiation
including the interactions of cosmic strings. Localization of individual GW
sources by the PTAs will be constrained to a few to tens of square degrees
\citep{EllisSiemens:2012}. Currently there are three active groups monitoring
pulsars for use as a GW detector, the Parkes Pulsar Timing Array
\citep[PPTA][]{PPTA:2013}, the European Pulsar Timing Array
\citep[EPTA][]{EPTA:2013}, and the North American Nanohertz Observatory for
Gravitational Waves \citep[NANOGrav][]{NANOGrav:2013}. The International
Pulsar Timing Array \citep[IPTA][]{HobbsIPTA+2010, ManchesterIPTA:2013} is a
consortium between these groups.






\subsection{The electromagnetic side}  
When black holes interact with gas and strong electromagnetic fields, they are
sources of bright EM radiation on their own (\textit{e.g.} active galactic
nuclei and x-ray binaries). Boasting surface fields of $\sim 10^{12}$ G and
up, neutron stars carry with them an enormous supply of potential EM energy
and are themselves observable within our Galaxy. It is thus plausible that, in
pairs, BHs and NSs could generate bright EM emission. Due to the potential
modulation of an EM signal from binary orbital motion, as well as extreme
energies that can be experienced at the end of the binary death spiral, such a
signature may not only be bright, but uniquely identifiable as well. The
benefit of EM signatures to GW events has been examined extensively in the
literature \citep[\textit{e.g.}][]{Bloom:EMWP:2009}, here we survey some of the key points.

If an electromagnetic signature can be identified with a GW event, we call it
an EM counterpart. Such an EM counterpart will allow localization of the GW
source on the sky, which as we saw above, is not easy to do with GWs alone.
Locating a bird by listening to its song vs. sighting it with your eyes, comes
close to the analogous problem of identifying a source location with multiple
gravitational wave detectors vs. pin-pointing its location with a telescope.
For LISA-like detectors, sky localization improves the
precision of the distance measurement, as this is largely limited by pointing
error \citep{Cutler:1998, Hughes:2002}. Localization will also give us contextual
clues to the nature of the source, constraining formation scenarios.

%Localization will also give us contextual
%clues to the nature of the source: is it in a galactic nucleus, at the heart
%of a globular cluster, or in the outskirts of a galaxy cluster? Such
%information could constrain formation scenarios.

If an EM observation yields not just a sky localization but also a redshift,
the corroboration of a GW measured distance and an electromagnetically
measured redshift can yield a precise measurement of the Hubble constant and
other cosmological parameters \citep{Schutz:1986, KrolakSchutz:1987,
ChernoffFinn:1993, Schutz:2002, HolzHughes:2005, Dalal:2006, Kocsis+2006,
Kocsis+2008, CutlerHolz:2009,  Nissanke:GRBStndSirens:2010, ShangHaiman:2011, Nishizawa:StndSirens:2011, Taylor:StndSirens:2012,
Tamanini:2016} as well as constrain fundamental physics such as the nature of
gravity on large scales \citep{Deffayet:2007, Camera:StndSirens:2013}.

EM counterparts can make independent measurements of binary parameters, removing
degeneracies in their determination \citep{HughesHolz:2003}, and they can be
used to reduce the signal to noise for GW detection \citep{KochanekPiran:1993,
HarryFairhurst:2011}. In general, EM counterparts are vital to determining the
astrophysical context of gravitational wave sources, allowing independent GW
and EM measurements to constrain models for EM emission \citep{Phinney:2009,
MandelO'Sh:2010}. The models for EM emission from BHNS mergers (Part II) could
soon be vetted in this way with LIGO observations.

EM signatures are useful even when they cannot be GW counterparts. There are
two types of EM signatures that are not counterparts. The first is based
purely on a practicality: an EM signature that would be a counterpart, but
cannot be because the source is out of the detectable distance or frequency
range of a detector (or the detector does not yet
exist). These EM signatures are useful in that they probe a missing part of
the population of sources and provide proof of existence in the case of
unbuilt instruments.
 %- if a tree falls in the woods, we could still see it! 
An example is gamma ray bursts (GRBs) that occur today outside of the LIGO volume
or any inspiraling MBHBs which occur this decade in the LISA band. These are
of course potential EM counterparts; given time and technology all such
sources are EM counterparts. \footnote{as long as they occur after the surface
of last scattering for photons -- the cosmic microwave background!}.

The second type of EM signature has no detectable GW counterpart by design.
But rather these fundamentally lonely EM signatures survey a part of a GW
source evolution before or after GW emission. For example, the early inspiral
of MBHBs \citep[\textit{e.g.}][]{Haiman+2008, HKM09} or the consequences of a BH
kick after merger \citep[\textit{e.g.}][]{Lippai:2008}. %, Rosotti:2012}. 
%
Each of these EM
signatures can allow us a glimpse into the broader evolution of the binary
system. The primary focus of Part I of this thesis is to make predictions for
the nature of EM signatures from the stage of MBHB evolution where the two
black holes are interacting with a gas disc. As we discuss in the next
section, this stage can overlap with a regime where the binary is emitting GWs
detectable by the PTAs and LISA, but the portion of inspiral before the binary
is in any GW band can provide unique EM identifiers of the binary which can
teach us about the environment of the central nucleus, the `final parsec
problem', and in general the MBHB path to coalescence.

The specific sources of GWs studied in this thesis are the inspiral and merger
of MBHBs in galactic nuclei and the merger of magnetized NSs with $\gsim 10
\Msun$ BHs. I now give background on each source in turn.






\section{Part I: Massive Black Hole Binaries} 
\subsection{Formation of MBHBs}     

The discovery of stellar mass BHs was textbook. In
the 1970's X-ray astronomy pioneers Ricardo Giacconi and Herbert Friedman lead
groups that carried out targeted searches for objects which consist of a
strong X-ray source orbiting a strong optical source, the X-ray source
thought to be emission from a BH accretion disc, fed by the optical source, a
star. Such systems had been envisioned from the theories of BH accretion and
stellar evolution. These predictions were confirmed observationally and called
the X-ray binaries, the first evidence of astrophysical BHs 
\citep[an entertaining historical account is found in][]{ThorneBHsTimeWarps:CH8}. 


The discovery of black holes millions to billions of times the mass of the
Sun, however, was not predicted outright, but was driven by observations of
the Quasars. The discovery of the Quasars and the realization that they must
be at cosmic distances, and so incredibly bright, forced theorists to predict
that their only plausible power source is the feeding of gas to massive BHs in
the heart of distant galaxies \citep[\textit{e.g.}][]{Schmidt:1963,
Salpeter:1964, LyndenBell:1969}. These arguments lead to our present day
understanding that a massive black hole (MBH) of $10^5 \rightarrow 10^{10}
\Msun$ resides at the heart of nearly every galaxy \citep{kr95,
KormendyHo2013, ff05}.


% The discovery of black holes millions to billions of times the mass of the
% Sun, however, came as a great surprise. It was not predicted that the enigma
% of the Quasars \citep[\textit{e.g.}][]{Schmidt:1963, Salpeter:1964,
% LyndenBell:1969} would lead to our present day understanding that a massive
% black hole (MBH) of $10^5 \rightarrow 10^{10} \Msun$ resides at the heart of
% nearly every galaxy \citep{kr95, KormendyHo2013, ff05}.

Further insight from cosmology adds to the story of MBHs. The hierarchical
formation of large scale structure, which is now standard lore of the $\Lambda
CDM$ cosmology, suggests that these MBH harboring galaxies merge \citep{HK2002}.
Indeed we see direct evidence of this in images of such mergers taking place
on the $\gsim 100$ kpc scale \citep{Comerford:2013}, \citep[see also][and
references therein]{Dotti:2012:rev}, as well as dual active galactic nuclei
(AGN) at the $\lsim 1$ kpc scale \citep{Komossa:2003, Fabbiano:2011,Rodriguez:2006, 
BurkeSpolaor:2011, ColpiDotti:2011:rev, Gitti:2013, Woo:subKpcBin:2014,
AndradeSantos:2016}


Based on the observations that galaxy centers harbor MBHs and galaxies merge,
the seminal paper by \cite{Begel:Blan:Rees:1980} first proposed that some
galactic nuclei may harbor two MBHs, and that these may form a massive
black hole binary (MBHB) which could eventually merge via emission of
gravitational radiation.

In this picture, the mass of the black hole and the cluster of gas and stars
which is bound to it will sink to the bottom of the new galactic potential via
dynamical friction \citep{Begel:Blan:Rees:1980, Chandrasekhar:1943}. Once the
separation of the binary is such that its binding energy is
greater than that of the surrounding star cluster, the binary is considered
hard, meaning that binaries with separation
\begin{equation}
a_h \lsim 2.8 \rm{pc} (1+q)^{-1} (1+1/q)^{-1}  \left( \frac{M}{10^8 \Msun} \right)  
\left( \frac{\sigma}{200 \rm{km s}^{-1}} \right)^{-2}
\end{equation}
can safely be treated as a Keplerian binary
\citep[\textit{e.g.}][]{MerrittMilos:2005:LRR}. Here $\sigma$ is the stellar
velocity dispersion of the nuclear star cluster and $q\equiv M_2/M_1$, with
the individual BH masses satisfying $M_2 \leq M_1$ and $M_1+M_2=M$.

Whether the binary becomes hard within a Hubble time depends on the mass ratio
of the binary, the amount of gas in the surrounding environment, and the
initial orbital parameters of the merger \citep{Mayer:2013:MBHBGasRev}. While
it is fairly certain that near equal mass galaxy mergers (with nearly equal
mass BHs) will quickly form hard MBHBs in less than a galactic dynamical
timescale \citep{Mayer+2007, Chapon+2013}, the case is not so clear cut for
disparate mass ratio mergers. If, for example, the mass ratio of merging BHs
and galaxies is 1:10, then it is possible for the tidal disruption time of the
smaller BH and its surrounding nuclear star cluster to be shorter than the
dynamical friction migration time. Because the dynamical friction timescale
scales inversely with the total mass of the BH and the matter bound to it
\citep{Chandrasekhar:1943, ColpiDotti:2011:rev}, such a scenario could leave
the smaller BH alone wandering naked in the galaxy \citep{Callegari:2011,
Mayer:2013:MBHBGasRev}. This is an example of how observations of MBHBs at
close separation, via GWs or the EM signatures discussed in this thesis, and
knowledge of their accretion history, will be vital in determining the
conditions that do (or do not) create disparate mass ratio binaries in
galactic mergers.

Once the MBHB hardens into a Keplerian binary, it must rely on stars which
come within $\sim3a$ of the binary to efficiently remove angular momentum and
cause further shrinkage \citep{Saslaw:1974}. However, in a closed (not
replenished),  spherical stellar system there are simply not enough stars on
centrophilic orbits to bring the binary to merger within a Hubble time. The
reason is that the mass in stars needed to merger the binary is of order a few
times the mass of the smaller BH \citep{MerrittMilos:2005:LRR}, but such stars undergoing this
`gravitational slingshot' mechanism are removed from orbits which can further
interact with the binary. Without a way to refill stars into the region of
energy-angular momentum space (the loss-cone) that allows nearly radial orbits
to interact with the binary, the binary stalls at a separation just below
$a_h$.

This situation has been dubbed the final parsec problem
\citep[FPP][]{Milosavljevic:2003:FPcP}. A number of ideas have been developed
to solve the FPP, including non-spherical stellar distributions which torque
stars into the loss cone over time, massive perturbers such as giant molecular
clouds, and the migration of the binary through a gaseous disc
\citep{GouldRix:2000, ArmNat:2002:ApJL, Mayer:2013:MBHBGasRev, Goicovic:2016}.
However the FPP is overcome (or not overcome), if the binary separation can
shrink to of order $0.05-0.15$ pc (for $q=1 \rightarrow 0.1$), then
gravitational radiation will take over and merge the binary within a Hubble
time \citep{Peters64}, generating the loudest sources of gravitational
radiation in the universe. This gravitational radiation will be a primary
target of the PTAs and eLISA both as individual events and as a stochastic
background.

In the work presented here we consider the case where the binary is surrounded
by an ample supply of gas in the pc to sub-pc regime. Torquing of gas to the
central regions of a galaxy is expected to occur during the galactic merger
process \citep{BH1992, Barnes:1996}, after which the gas can cool and form a
disc \citep{Barnes:2002}. This gas could be important for solving the FPP and
altering binary parameters, affecting GW waveforms near merger
\citep[\textit{e.g.}][]{ArmNat:2005, YKH:2011:L, RoedigSesana:2012:eccGWs},
%
%Interaction with gas is also important for determining the rate of GW events
%detectable by LISA and the PTAs, 
%DD may not be important for LISA
and determining the MBHB stochastic GW
background \citep{KocsisSesana:2011, Shannon:2015,
Sesana+2016:GWB, EPTA:GWB:2015, Arzoumanian:2015:SGWB}. In addition to its
importance for orbital dynamics, the gas surrounding a hard MBHB will be vital
for creating unique EM signatures of the binary during early inspiral (Part I
of this thesis), merger \citep{Chang:2010, Baruteau:2012,
CerioliLodato:Squeeze:2016} and post merger \citep{Lippai:2008, Lia10,
Rossi:2010, Ponce:2012, Rosotti:2012, Zanotti:2012} via accretion and shocks.
Understanding these potential EM signatures, requires knowledge of the binary
and disc interaction, a rich topic which we now review.




\subsection{Interaction with a gas disc}

The interaction of a gas disc and a binary has been studied extensively in the
astrophysical literature as it manifests in a large variety of systems. These
include proto-planetary discs \citep[\textit{e.g.}][]{Ward:1997}, young binary
star systems \citep[\textit{e.g.}][]{AL94}, the rings around planets
\citep{GTSaturn78}, and AGN scale discs surrounding MBHBs
\citep[\textit{e.g.}][]{GouldRix:2000}. Though the physics describing tidal
coupling between gas and binary is the same in each case, the specifics of
scale can differ in an important way.  An example directly relevant to the
work in Part I of this thesis, is the mass ratio of the binary. In the case of
planets and planetary rings, the secondary body (the smaller mass planet, or
the moon in a planetary ring) is much smaller than the primary body (the star
or, the ringed planet) because it formed from the leftovers of the primary. In
the case of our solar system, the planets grew out of a protoplanetary disc
with total mass much less than that of the Sun; the Sun-Earth mass
ratio is $\sim10^{-6}$ while the Sun-Jupiter mass ratio is $10^{-3}$. Binary
star systems and MBHB systems, however, have the propensity to form with mass
ratios closer to unity \citep[for MBHBs see previous section, for stars see][]{StellMassRatDist:2015}). This variation in typical mass ratio across
systems results in drastically different expected behavior of the binary and
disc in each system. Here we introduce the typical regimes as they vary with
binary mass ratio.

%Gas cool into thin disc:  Escala et al. 2005
%%%%%%%%%%%%%%%%%%%%%%%%%%%%%%%%%%%%%%%%%%%%%%%%
%%% FIGURE: 2D density%%%
%%%%%%%%%%%%%%%%%%%%%%%%%%%%%%%%%%%%%%%%%%%%%%%%
\begin{figure}
\vspace{-0.0in}
\begin{center}$
\begin{array}{c c c}
%
\includegraphics[scale=0.294]{figures/ch0/TypeI_Duffell}& \hspace{-0 pt}
%
 \includegraphics[scale=0.4]{figures/ch0/JupiterTypeII} &  \hspace{-0 pt}
%
 \includegraphics[scale=0.4]{figures/ch0/q1_CentralCav} 
\end{array}$
\end{center}
%\vspace{-0.35in}
%\caption{Snapshots of density from 2D hydrodynamical simulations for binaries on a fixed circular orbit, with increasing binary mass ratio from left to right ($q=10^{-6}, 10^{-3}, 1.0$). The Left panel is adopted from \citep{DM2012:gaps} and depicts the linear, Type I regime, the middle panel depicts the Type II regime, and the right panel.} %
\caption{The left panel is for a binary with $q\equiv M_2/M_1 =10^{-6}$ (adapted from \citep{DM2012:gaps}), such a small secondary excites linear spiral density waves in the disc causing Type I inward migration of the binary. The middle panel is for a binary with $q=10^{-3}$. The dark ring in the orbit of the smaller black hole is the low density gap synonymous with Type II migration. The right panel depicts the clearing of a central (time-fluctuating), low density cavity around an equal mass binary.}
\label{Fig:IntroHydro}
\vspace{-0.2in}
\end{figure}
%%%%%%%%%%%%%%%%%%%%%%%%%%%%%%%%%%%%%%%%%%%%%%%%


\cite{LinPapa79a, LinPapa79b}, \cite{GT79}, and \cite{GT80} laid the
groundwork for disc interactions with very small mass ratio systems, where the
response of the binary and disc can be explored with linear perturbation
analysis. In this case, the secondary launches linear spiral density waves
from the locations of Linblad resonances in the disc
\citep{LyndenBellKalnajs:1972}. Summing contributions from torques exerted on
the disc at these resonances, \cite{GT80} were the first to show that the
back-reaction of the disc perturbations onto the binary cause the binary
orbital separation to change. For discs with Keplerian rotation profiles,
inward torques on the secondary from the outer Linblad resonances outweigh the
outward torques on the secondary from the inner Linblad resonances, and inward
`migration' (orbital shrinkage) occurs \cite{Ward:1986}. This process, where
linear spiral density waves are launched by the secondary and cause the
binary's orbit to shrink, is called Type I migration 
\citep[See also][]{MVSic87, Ward:1997, TanakaI:2002, TanakaII:2004}. 
Hence, the solutions to the equations of hydrodynamics, for discs perturbed by
a small mass ratio binary, consist of wave solutions launched from the
position of the secondary. In the frame of the binary, these waves have a
stationary phase and once they propagate into the disc on both sides of the
binary, the disc approaches a static solution which follows the secondary
component as it slowly changes its orbital radius and possibly eccentricity
\citep{GT80, Ward:1988, GoldreichSari:2003}. This steady spiral density wave solution is depicted in the left panel of Figure \ref{Fig:IntroHydro}.





When the binary mass ratio is large enough, the spiral density waves launched
in the disc become non-linear at a short distance (less than a disc scale
height) from the secondary \citep{GoodmanRafikov:2001}. The waves steepen into
a shock and deposit angular momentum to the disc material in the co-orbital
region of the secondary \citep[see Chapter \ref{ch:CBDTrans} and also][]{DongRafI:2011, DongRafII:2011, LinPapaI:1984,LinPapa86b,
LinPapaIII:1986}. This process clears a low density annulus in the orbit of
the secondary and is depicted in the middle panel of Figure
\ref{Fig:IntroHydro}. \cite{LinPapa86b} argued that if
such a gap is formed, the secondary will be locked into the radial flow of the
disc, migrating at the viscous inflow rate. Such migration, when a gas barrier
is formed around the binary is called Type II migration \citep[see also][and Chapter \ref{ch:CBDTrans}]{Ward:1997, KleyNelson:2012:rev}



Besides the mass ratio, an important difference between different binary+disc systems is the
total gas reservoir. Analytical work by \citep{SyerClarke95, Ivanov99}, in
one dimension,\footnote{averaging disc height and azimuth} showed that in the
non-planetary case, the Type II rate would eventually slow on scales where the
mass of the disc becomes smaller than the mass of the migrating binary
component. The argument being that the gas no longer has a large enough angular
momentum reservoir to shrink the binary separation on the viscous
timescale. Hence, this `secondary dominated migration' would cause a pileup of
gas behind the secondary and the gas interior to the secondary's orbit would
drain onto the primary creating a central cavity devoid of gas and possibly
halting accretion onto the binary. Other 1D arguments
\citep[][]{Milos:Phinney:2005} and even early 2D smoothed particle hydrodynamics
(SPH) simulations \citep{Artymowicz:1991} concluded that the outward torques
from the binary would clear a cavity around most binary systems in the Type II
regime.


This picture, while laying the groundwork, has been greatly altered by work in
the intervening two decades, notably by the advent of two-dimensional
numerical, hydrodynamical simulations which capture the full non-axisymmetric
nature of the binary disc interaction, and allow global, time-dependent
solutions. The first of such numerical calculations was carried out in
\cite{AL94} and \cite{ArtyLubow:1996} who ran SPH simulations to test analytic
work that predicted the sizes of circumstellar discs in binaries and the sizes
of the central cavities surrounding the stellar binaries. These SPH
simulations showed that particles, in the form of streams tidally ripped from
the edge of the cavity wall, could indeed flow past the binary tidal barrier
and reach the binary components. The ability of gas to flow past the tidal
barrier is of two-fold importance. First, it can allow high levels of
accretion onto the binary, which could generate a bright EM signature of the
binary, and second, it affects migration (and hence merger) rates of binaries
in gas discs \citep[\textit{e.g.}][]{Kocsis+2012a, Kocsis+2012b, Rafikov:2013,
Rafikov:2016}. The implications of both are currently areas of active
research.




This above 1D studies also naturally fail to account for mass flow across the
gap along horseshoe orbits in the full dimensionality of the problem. Recent
work, using 2D viscous hydrodynamical simulations has shown that mass flow
across the gap, can allow the secondary to migrate at a rate dependent on disc
parameters (density, temperature, pressure), and limited by a maximum migration
velocity which can be greater than the viscous rate \citep{Edgar:2008,
DuffellFTV:2014, DurmannKley:2015}. The mechanisms which dictate the migration
rate of gap opening planets in the full two and three dimensional pictures is
a topic of ongoing work.


% Additionally Chapter \ref{ch:CBDTrans} of this thesis
% \citep{D'Orazio:CBDTrans:2016} shows that the clearing of a central cavity is
% not necessarily due to secular effects as in the picture of disc dominated
% migration \cite{SyerClarke95}. Chapter \ref{ch:CBDTrans} provides evidence
% that the clearing of an annulus in the orbit of the secondary gives way to a
% much more violent clearing of a \textit{central cavity} for mass ratios above $q
% \sim0.04$.

Additionally, chapters \ref{ch:CavAcc} \citep{DHM:2013:MNRAS} and \ref{ch:CBDTrans}
\citep{D'Orazio:CBDTrans:2016} show that the Type I to Type II regimes are not
the only that depend on mass ratio. For binary mass ratios above $q\sim0.04$,
a mass ratio well into the Type II regime for thin discs, the clearing of an
annulus in the orbit of the secondary gives way to a much more violent
clearing of a lopsided, \textit{central cavity} and time dependent behavior (see the right panel of Figure \ref{Fig:IntroHydro}).
From mass ratios $q \gsim 0.3$ the lopsided central cavity is highlighted by an
orbiting overdensity at its inner wall. Chapters \ref{ch:CavAcc} and
\ref{ch:CBDTrans} provide more details on these transitions and their
importance for observing MBHBs.




The work in this thesis focuses on the implications for accretion onto the binary. Hence we now summarize the recent work on this front.


\cite{Hayasaki:2007} conducted the first 3D-SPH simulations that specifically
targeted MBHB systems with the intent to measure accretion rates onto the
binary. \cite{Hayasaki:2007} ran simulations of binaries with mass ratios
$q=1.0$ and $q=0.5$ and binary eccentricities $e=0.0$ and $e=0.5$ for up to 60
binary orbits. They find that streams are indeed pulled into a central, low
density cavity forming a triple disc system \citep{Hayasaki+2008} consisting of
the circumbinary disc and mini-discs around each binary component. The streams
promote accretion onto the BHs at rates as high as a tenth of the Eddington
rate. For eccentric binaries only, \cite{Hayasaki:2007} found a strong modulation in
the accretion rate at the binary orbital period. 
%Cons - run for a short time at low resolution, injection of particles at r=1.65 seems artificial

The SPH simulations of Hayasaki were soon succeeded by the 2D, grid based 
simulations of \citep[][hereafter MM08]{MacFadyen:2008},  run for 1000's of binary orbits (greater than a
viscous time at the position of the binary). These higher resolution
simulations, using the FLASH code \citep{Fryxell:2000}, are more adept at
capturing supersonic dynamics in the vicinity of the binary (shocks). Though
MM08 cut out the inner region of the domain containing the binary, they
measure accretion rates into the inner boundary which is inside the low
density central cavity set by the initial conditions. The high resolution
simulation of MM08, for an equal mass binary, found new behavior: the
elongation of the central cavity which results in high levels of accretion
into the inner simulation boundary. The resulting periodogram of the
accretion rate has the strongest peeks at a low frequency corresponding to
$4.5 \times$ the binary orbital period and at a frequency corresponding to
twice the binary orbital period. Though not discussed in MM08, the cause of
accretion variability at these timescales is elucidated in Chapter
\ref{ch:CavAcc} of this thesis and also \cite{ShiKrolik:2012} below.


Further SPH studies of MBHB systems were conducted by \cite{Cuadra:2009} who
ran 3D simulations at a resolution $10 \rightarrow 100$ times higher than that of
\cite{Hayasaki:2007} for marginally self-gravitating discs with a binary mass
ratio of $q=0.3$ and a simple cooling prescription for the gas. They do not
find elongation of the cavity as in MM08, though this could be due to the
short amount of time for which the simulations are run, $\sim 200$ orbits, or
the resolution loss that SPH simulations suffer in low density regions (namely
the dynamically import cavity edge of the circumbinary disc).
\cite{Cuadra:2009} do find an accretion rate variable at the orbital period
and a propensity for the gas disc to excite binary eccentricity.
\cite{Cuadra:2009} also finds that the secondary BH has a larger ($\times 2$)
accretion rate than the primary due to its closer proximity to the edge of
the central cavity.


\cite{Roedig:2012:Trqs} carry out similar simulations to \cite{Cuadra:2009}, except they start the binary at different initial eccentricities $e_0$ finding that eccentricity damps for $e_0 \gsim 0.6$ but is excited for $e_0 \lsim 0.6$, suggesting the existence of a rather large preferred binary eccentricity. \cite{Roedig:2011:eccevo} consider different disc thermodynamics and different accretion (sink) prescriptions. In both cases the accretion rates onto these eccentric binaries are found to have periodicity at the binary period and its harmonics, but also at lower frequency disc periods and beat frequencies between disc and binary periods.


The first magneto-hydrodynamical (MHD) simulations of the circumbinary disc
were carried out by \cite{ShiKrolik:2012} with a grid based code.
\cite{ShiKrolik:2012}  performed both 2D hydrodynamical and 3D MHD simulations
of an equal mass binary on a circular orbit with a similar setup to
MM08. Despite a higher overall accretion rate due to larger
viscous stresses generated by the Magneto-rotational instability (MRI),
\cite{ShiKrolik:2012} find similar results to MM08, in that they also find the
growth of a lopsided central cavity (elongated with a cavity wall
overdensity), which generates variable accretion into the central simulation
domain. The variability of the accretion rate is in agreement with MM08,
exhibiting a long period variation at the period of gas orbits at the cavity
wall, and a second period at twice the binary orbital period (due to the
symmetry of an equal mass binary sweeping through the near side of the lopsided
cavity). \cite{ShiKrolik:2012} provide evidence that the cavity lopsidedness is
due to the kinematics of stream impacts and recycling of the cavity wall
overdensity: the cavity wall overdensity periodically shears apart, causing a
lump to orbit around the cavity, feeding streams which are flung out of the
cavity again to generate the cavity wall overdensity. \cite{ShiKrolik:2015}
have extended upon the above work by considering a range of binary mass ratios,
finding qualitative agreement with \cite{DHM:2013:MNRAS} and \cite{Farris:2014}
discussed below.

MHD simulations by \cite{Noble+2012} incorporate post-Newtonian corrections
to the disc hydrodynamics and binary orbital decay in order to track the disc
response through binary inspiral. \cite{Noble+2012} find that gas can follow
the binary down to separations of $\sim10M$ with only a $\sim10 \rightarrow 20\%$
reduction in accretion rate. They also find a lopsided central circumbinary
disc cavity, in agreement with MM08, \cite{ShiKrolik:2012}, and the works
that we discuss next.

Chapter \ref{ch:CavAcc} of this work \citep{DHM:2013:MNRAS}, extends the work
of MM08 (using the same numerical code and a similar numerical setup) by
considering not only equal mass binaries but a range of binary mass ratios
from $q=0.01 \rightarrow 1$. For an equal mass binary, the qualitative results
of disc response and accretion rate variability found in MM08 and
\citep{ShiKrolik:2012} are reproduced and compared to the magnitude of
accretion for a point mass. By varying $q$, however, a landscape of accretion
variability and magnitude is uncovered and discussion of its use for MBHB
searches is discussed.

\cite{Farris:2014} extended the work of \cite{DHM:2013:MNRAS} by adapting the
moving mesh code DISCO \citep{Duffell:2011:TESS,
DuffellMHDDISCO:2016} to track, for the first time, gas dynamics in the
vicinity of the binary using a grid based (rather than an SPH)
code. \cite{Farris:2014} finds results in agreement with MM08 and
\cite{DHM:2013:MNRAS} and finds also that the relative accretion rate onto
each black hole is a function of mass ratio, dominated by the secondary from
$0.05 \leq q < 1$. 


For the equal mass case \cite{Farris:2015:GW} considered the effects of
gravitational wave decay on the circumbinary disc system showing that gas
could indeed follow the binary to small separations causing variable accretion
up until merger, contrary to previous lore that gas should be left behind in a
`decoupling phase' by a binary that is quickly merging due to GW emission (see
\ref{ch:PG1302_a} for further context).\footnote{though this result may be
dependnent on disc parameters} Finally \cite{Farris:2015:Cool} implemented a
simple cooling prescription in DISCO (previous work being for isothermal
discs) showing that the variability of the accretion luminosity should indeed
follow what was predicted in previous works for the variability of the
accretion rate. These results have been key to applying accretion rate
variability predictions to the search for close MBHBs discussed in the next
section.



Recent work has examined the nature of gas temperature on accretion rates.
Both \cite{YoungClarke:2015} and \cite{RagusaLodato:2016} use SPH codes (2D
and 3D respectively) to simulate a range of binary mass ratios above $q=0.1$
and vary the gas temperature. In these simulations, the gas temperature
manifests in the form of the disc vertical height to radius aspect ratio,
$h/r$ which, in vertical hydrostatic equilibrium, is equal to the ratio of the
sound speed to the gas angular orbital frequency at distance $r$ from the
system barycenter. A thicker disc, is hotter and has larger pressure forces.
Both studies find that, while simulations of accretion onto MBHBs (using $h/r \sim
0.1$) accrete at near the value for a single BH, more realistic, colder AGN
discs ($h/r \sim 0.01$) should accrete at much lower rates. Though
interesting, the robustness of these results remains to be seen as numerical
difficulties arise in cold discs.%, especially when using SPH codes \citep{}.


Notably, the simulations of \cite{RagusaLodato:2016} capture the lopsided disc
behavior with a circular binary. Except for a study which considered an
eccentric binary \citep{Dunhill+2015}, no SPH simulations have captured the
lopsided disc behavior. It is not yet clear however what has allowed this
change. The SPH works to date are computed with different numerical codes, at
different resolutions, and for different total numbers of binary orbits.

The first application of an arbitrary shaped moving mesh code \citep{AREPO}
was recently applied to the problem of binary+disc interaction for circular
and eccentric stellar binaries by \cite{MunozLai:2016}. This work simulates an
equal mass binary and finds lopsided cavities and accretion rate variability
at the long term period associated with an orbiting cavity wall overdensity
and also variability at twice the orbital frequency, in agreement with
\cite{DHM:2013:MNRAS}, \cite{Farris:2014}, \cite{ShiKrolik:2012}, and
\cite{ShiKrolik:2015}. For eccentric binaries, \cite{MunozLai:2016} find that
the the long-term period is overwhelmed by orbital timescale periodicity and
that even for an equal mass binary, one component will accrete more than the
other on timescales set by the precession of the lopsided circumbinary disc
\citep[see also][]{Dunhill+2015}. Implications of the existence of the long
timescale periodicity are discussed in Chapter \ref{ch:PG1302_a}.

% \cite{MunozLai:2016} do find a much closer match in the time series accretion rates onto each component of the binary. \citep{Farris:2014} finds that the average accretion rates to match for the two components of an equal mass binary, but there exact time series wander. It may be that the use of an adaptive moving mesh by \citep{MunozLai:2016} is crucial for capturing accurate accretion rates.


In addition to prograde discs in the plane of the binary, some groups have considered retrograde discs \citep{Nixon:2011:LongSim, RoedigSGmigrate:2014, DunhillNixon:2014, BankertShiKrolik:2015, NixonLubow:RetroRes:2015, AmaroSeoane:RetroDiscs:2016} and the alignment or tearing of warped discs \citep{NixonKingPringle:2011, Nixon:CntrAlign:2012, Hayasaki:misalignSims:2013, NixonKing:Tear:2013, DoganNixonKingPrice:2015, Goicovic_I:2016}.

Also MHD simulations in full general relativity have been carried out
by \cite{FarrisLiuShap:2010:Bondi, FarrisShap:2011, FarrisGold:2012,
Gold:GRMHD_CBD:2014, Gold:GRMHD_CBDII:2014} in the regime just before merger,
showing also that accretion rates can be of order the rate expected onto a
single BH, periodic, and the gas can follow the binary down to separations of
order a few $M$, allowing the binary to be bright up until merger.





%%%Observations
\subsection{Observations of MBHBs}

%%%%%%%%%%%%%%%%%%%%%%%%%%%%%%%%%%
%%%FIGURE doppler candidates
%%%%%%%%%%%%%%%%%%%%%%%%%%%%%%%%%%
\begin{wrapfigure}{R}{0.5\textwidth}
%\begin{figure}
\begin{center}
\includegraphics[scale=0.55]{figures/ch0/PTF_xsiGtr1_vs_z_M_P_alph_I90} 
\end{center}
\caption{
A subset of the MBHB candidates from \citep{Graham+2015b} and \citep[][denoted by black x's]{Charisi+2016} 
for which spectral slopes are measured and the
magnitude of variability from Doppler boosting can be estimated. From left to
right, top to bottom, the ratio of predicted Doppler variability amplitude to
observed variability amplitude is plotted vs. redshift, average optical
magnitude, log binary mass, and observed period. Candidates above the
horizontal black line are possible Doppler boost MBHB candidates.
}
\label{Fig:DopCan}
%\end{figure}
\end{wrapfigure}
%%%%%%%%%%%%%%%%%%%%%%%%%%%%%%%%%%%%%%%%%%%%%%%%


A motivation for the above theoretical calculations is to
determine the types of EM signatures that will identify MBHBs in the inspiral
regime. Searching for MBHBs by searching for periodically varying AGN has been
proposed before by MM08, \cite{Haiman+2009}, and by HKM09. \footnote{A number
of other methods for identifying MBHBs with EM signatures exist in the
literature, \textit{e.g.}, spectroscopic signatures of a circumbinary accretion
disc, peculiar radio morphology, and broad emission line shifts. For a description of
MBHB candidates found through these complimentary methods, see the
introduction of \cite{Charisi+2016}.}
 

HKM09 propose that close MBHBs can be identified in Quasars by their
production of EM emission modulated at the binary orbital period. Under this
assumption they compute the duty cycle of MBHBs with periods observable in
human lifetimes by computing the residence times of MBHBs at a given orbital
period (binary separation) taking into account gas induced migration and also
GW driven inspiral. Comparison of the residence time to the average Quasar
lifetime allows HKM09 to predict the solid angle, depth, and cadence of an EM
time domain survey required to capture a specified number of MBHBs at a given
orbital period and luminosity.

Although a number of single objects have been identified as MBMB candidates
through EM variability (\textit{e.g.} OJ 287 \citep{LehtoValtonen1996} and the
tidal disruption candidate SDSS J120136.02+300305.5 \citep{LiuKomossa:2014}),
large systematic surveys capable of searching 10's to 100's of thousands of
AGN for EM variability came to fruition only a year ago. A group from
Caltech/JPL scoured 9 years of time-domain, optical photometry of
$\sim250,000$ quasars in the Catalina Real-Time Transient Survey
\citep[CRTS][]{CRTS1:Drake:2009, CRTS2:Djorgovski:2010, CRTS3:2011Mahabal,
CRTS4:Djorgovski:2011} attempting to characterize quasar variability. They
found a subset of periodically varying sources. The brightest of these sources
is PG 1302-102 which was identified as a close, $a \sim 0.01$pc separation
MBHB candidate by its nearly sinusoidal variability in the V-band continuum.
PG 1302 is the first MBHB identified in this manner, \footnote{Though the MBHB
candidate OJ 287 is identified by repeating (though not periodic) flares from
over a century of data \citep{LehtoValtonen1996, Pursimo:2000}} and had the
closest reported binary separation at the time of publication
\citep{Graham+2015a}. The second part of Part I of this thesis uses the
theoretical developments of the first part to interpret the binary candidate
PG 1302, finding that PG 1302 is most likely described by a system with a
disparate mass ratio where the smaller BH is emitting most of the optical
light that is modulated via relativistic Doppler boosting.


Soon after the announcement of PG 1302, 110 more MBHB candidates, were picked
out of the CRTS for their periodic optical light curves \citep{Graham+2015b}
and then 33 more, at shorter periods \citep{Charisi+2016}, from the Palomar
Transient Factory \citep[PTF][]{Rau:2009, Law:2009}. Figure \ref{Fig:DopCan}
displays a subset of candidates for which Maria Charisi and I have measured
the expected maximum amplitude of optical variability due to Doppler Boosting.
Comparing to the observed amplitude of variability, Figure \ref{Fig:DopCan}
plots the sample versus various population characteristics and delineates the
fraction of the sample which could be caused by the Doppler boost model (see
Chapter \ref{ch:PG1302_b}). It is interesting to note that $\sim \% 80$ of the
candidates have large enough maximum orbital velocities \footnote{maximum
refers to an assumption of an edge on binary inclination and $q \rightarrow
0$} to account for their optical variability by relativistic Doppler boost
alone. Because smaller mass ratio binaries are preferred for Doppler Boost
candidates (Chapter \ref{ch:PG1302_b}), it is also interesting to note that
\citep{Charisi+2016} find the PTF MBHB candidates to be consistent with a
population of low ($q\sim0.01$) mass ratio binaries. The characterization of
this population of MBHB candidates will be very interesting to track in the
near future.


In addition to these discoveries, a few single candidates have been announced
from time domain periodicity arguments: 

\cite{Liu:7RsMBHB:2015} report the detection of a $\sim 541$ day periodicity in
the g, r, i, and z bands of the PAN-STARRS1 medium deep survey. At redshift
$z=2$, and with a measured total binary mass of $\sim 10^{10} \Msun$, this
puts the putative MBHB candidate at a separation of $\sim10$ Schwarzschild
radii. Though such a find is extremely unlikely given the short residence
times at this separation (HKM09), the hypothesis will be testable as the binary
period should speed up as GWs bring the binary to coalescence in the next
$\sim 7 (1+z)$ years!


\cite{Zheng:MBHB_2P:2015} report a MBHB candidate in SDSS J0159+0105 with a
centi-parsec separation at $z=0.217$. This interesting object was found in the
CRTS, but not with a single period, as the \cite{Graham+2015b} search was
likely most sensitive to, but with periods at a 2:1 ratio (741 and 1500
days), a characteristic of the simulations of ($q \neq 1$) circumbinary
accretion presented here and in other works discussed above.


\cite{LiWang:2016} find evidence for a centi-parsec separation MBHB in the
center of NGC 5548. They determine a 14 year orbital period from the optical
variability in conjunction with reported orbital variations in the H$\beta$
emission line on the same timescale.



Follow up observations are needed to secure the nature of these candidates.
Quasars exhibit intrinsic, wavelength dependent variability
\citep{Kelly:2009:DRW,Kozlowski+2010} and it must be confirmed whether or not
the observed periodicities are random manifestations of this intrinsic
variability. As Jules Halpern says: `periodicity is the easiest thing to prove
in astronomy, you just have to wait'. However, further evidence, across
wavelengths can help pin down the mechanisms driving such periodicity,
possibly ruling out models, binary and not, for the production of periodic
emission in quasars. Work must be done to place MBHBs in their full gassy,
dusty environments in active galactic nuclei. Then we can begin to piece
together a multi-wavelength portrait of MBHBs and distinguish them amongst the
single BH quasars. The final chapter of Part I (Chapter \ref{ch:Dust}) is a
beginning to this process. In Chapter \ref{ch:Dust}, we present a model for
the infrared variability expected from dust reverberation by MBHBs that
exhibit variable emission, through either accretion variability or anisotropic
Doppler boosted emission.




































\section{Part II: Stellar Black Hole + Neutron Star Binaries}

The merger of NSs and stellar BHs will generate GWs detectable by the Laser
Interferometer Gravitational-Wave Observatory \citep[LIGO][]{aLIGO:2015}.
Binaries with BHs will generate the highest amplitude GW signals \textit{e.g.}
Eq. (\ref{Eq:BinStrain}), but a binary containing a NS has the most potential to
produce a bright EM signal, making BHNS systems especially interesting sources
of EM+GW emission.


%%%%%%%%%%%%%%%%%%%%%%%%%%%%%%%%%%
%%%FIGURE BHNS TDs
%%%%%%%%%%%%%%%%%%%%%%%%%%%%%%%%%%
\begin{wrapfigure}{R}{0.4\textwidth}
%\begin{figure}
\begin{center}
\includegraphics[scale=0.33]{figures/ch0/BHNS_TDs} 
\end{center}
\caption{Approximate values of black hole mass and spin for which a companion neutron star would be swallowed whole (shaded) vs. disrupted outside of the black hole horizon (unshaded). The two shaded regions are for the labeled neutron star masses, spanning the range of theoretical limits, and for a neutron star with radius 10km.}
\label{Fig:NSBH_TDs}
%\end{figure}
\end{wrapfigure}
%%%%%%%%%%%%%%%%%%%%%%%%%%%%%%%%%%%%%%%%%%%%%%%%

The tidal disruption of a NS by its BH partner could generate a $\gamma$-ray
burst after merger \citep{NPP:NSBH_GRB:1992}. However, it is under-appreciated
that most BHs should be large enough to swallow their NSs
whole, causing the mergers of most BHNS binaries to be dark. Figure 
\ref{Fig:NSBH_TDs} plots the simplest approximation for the disruption condition,
\begin{equation}
 r_T \approx \left( \frac{M_{\rm{BH}} }{ M_{\rm{NS}}} \right)^{1/3} R_{\rm{NS}} \geq r_H(S) 
 = M_{\rm{BH}} + \sqrt{M^2_{\rm{BH}} + S^2},
 \label{Eq:rT}
\end{equation}
which requires that the disruption radius $r_T$ be outside of the BH event horizon
with dimensionless spin $S$ (using natural units for the BH horizon
radius $r_H$). Figure \ref{Fig:NSBH_TDs} shows that, unless the BH has near maximal
spin, BHNS systems with $M_{\rm{BH}} \gsim 6 \msun$ will swallow the NS whole!
Eq. (\ref{Eq:rT}) is of course a crude approximation which depends on the
(unknown) equation of state of the NS. More sophisticated approximations,
however, do not find anything drastically different
\citep[\textit{e.g.}][]{Foucart:2012}. Furthermore, the predictions for EM
signatures of non-disrupting BHNS mergers, will be necessary for learning
about the NS equation of state once coincident GW observations can be made.


Although the distribution of BH masses which will merge with
a NS is unknown, it is interesting to note that the BH mass distribution
inferred from BHs in X-ray binaries peaks around $8 \Msun$ \citep{Ozel:2010}
and the only known BH binary consisted of BHs with masses $\sim 30 \Msun$, which
would certainly swallow a NS hole \citep{GW150914:2016}. Though suggestive, it is important to keep
in mind that each of these formation channels may be independent, and not
applicable to a BHNS system.

As additional motivation, LIGO is the most sensitive at a frequency of $\sim
200$ Hz, this is the gravitational wave frequency at coalescence for a NS of
mass $1.4 \Msun$ in a circular orbit with a BH of mass few $\sim100
\Msun$. If such binaries occur in nature, they have the potential to be high
signal to noise LIGO detections, and will not disrupt the NS. The
above motivates an exploration of EM counterparts to non-disrupting BHNS
systems.


A possible pathway for bright EM emission by non-disrupting BHNS mergers is
through the electromagnetic interaction of the NS magnetosphere and the BH
event horizon. In such an interaction, the BH horizon behaves like a conductor
\citep[see][and Chapter \ref{ch:Rindler}]{MPBook}, spinning and moving through
the magnetic fields of the NS. The generation of EM radiation from similar
situations, of a conducting body moving through the magnetic fields of
another, has been investigated in application to a number of other
astrophysical systems, \textit{e.g.} Jupiter and its moon Io \citep{GLB:1969},
planets around white dwarfs \citep{Li:1998} and main sequence stars
\citep{LaineLinI:2012,LaineLinII:2012}, binary neutron stars
\citep{Vietri:1996,Piro:2012, DLai:2012, Palenzuela:2013}, compact white dwarf
binaries \citep{Wu:2002, Dall'Osso:2006, Dall'Osso:2007, DLai:2012}, BHs
boosted through magnetic fields \citep{Lyut:2011, Penna:2015}, and the
Blandford-Znajek (BZ) mechanism \citep{BZ:1977} for a single BH spinning in a
magnetic field  \citep[for recent numerical work on the BZ mechanism see \textit{e.g.}][]{PalenzuelaBZ:2011, Kiuchi:2015}.

%The calculation for BHNS systems, 
%already presented in Ref.\ \cite{McL:2011}
%and confirmed in the detailed relativistic analysis of
%Ref.\ \cite{DorazioLevin:2013}, as well as the numerical calculations
%of Ref.\ \cite{Paschalidis:2013}, gives the scaling of power available
% for conversion into electromagnetic luminosity. In the next section we
% will consider the implications of throwing this power into luminous
% elements in the BHNS circuit.

\subsection{An analogy from Faraday}
To introduce this mechanism, I want to first introduce a similar, though
subtle example of the Faraday disc. The Faraday disc is a type of unipolar
inductor constructed by placing a conducting rod through the center of a
conducting disc, and running a wire from the top of the rod to the outer edge
of the disc, where a sliding contact completes a circuit (see Figure
\ref{Fig:FDschem}). Tracing a magnetic field perpendicularly through the disc,
and spinning the disc generates an electromotive force (emf), $\xi$. We can compute
the voltage drop from the center of the disc to the edge of the disc from
Faraday's law
\begin{align} 
\xi = - \frac{1}{c}\frac{d}{dt}\int_{\Sigma(t)}{\Bvec \cdot d\Avec}, 
\end{align} 
where the circuit bounds an open, time-dependent surface $\Sigma(t)$. At first
glance, it seems that the emf should be zero, as the obvious loop (loop $a$ in
Figure \ref{Fig:FDschem}) connecting wire to disc to rod has zero magnetic
flux. However, Faraday discs do generate an emf, and this is easily verified
by considering the Lorentz force on electrons. To see this from Faraday's law, recall two restrictions in the
choice of the open surface of integration $\Sigma(t)$. One: $\Sigma(t)$ must be bounded by the closed loop through which the emf is computed, and Two: $\Sigma(t)$ must capture the relative motion of the circuit.
%\begin{enumerate} 
%\item $\Sigma(t)$ must be bounded by the closed loop through which the emf is
%computed. 
%\item $\Sigma(t)$ must capture the relative motion of the circuit.
%\end{enumerate}
The key is in the second point: the part of the circuit that starts in the
disc must move along with the spinning disc, otherwise you implicitly assume
that the sliding contact and the disc are not in relative motion - but they
are by construction.


%%%%%%%%%%%%%%%%%%%%%%%%%%%%%%%%%%
%%%FIGURE Faraday disc
%%%%%%%%%%%%%%%%%%%%%%%%%%%%%%%%%%
\begin{wrapfigure}{R}{0.4\textwidth}
%\begin{figure}
\begin{center}
\includegraphics[scale=0.33]{figures/ch0/UI_schematic} 
\end{center}
\caption{Schematic of a Faraday disc (unipolar inductor).}
\label{Fig:FDschem}
%\end{figure}
\end{wrapfigure}
%%%%%%%%%%%%%%%%%%%%%%%%%%%%%%%%%%%%%%%%%%%%%%%%


To calculate the emf, choose loop $b$ in Figure \ref{Fig:FDschem} which moves
along at the rate of the spinning disc, $\Omega = d\phi/dt$. Say that the
radius of the disc is $R$ and the uniform magnetic field tracing the disc is $\Bvec$, then, working in polar coordinates $(r, \phi)$,
\begin{align}
\xi_{\rm FD} &= - \frac{1}{c} \frac{d}{dt} \int_{\Sigma(t)}{\Bvec \cdot d\Avec} =  - \frac{1}{c} \int^{\phi(t)}_{0} \int^{R}_{0}{B r dr d\phi}  \\
&= - \frac{1}{c} \int^{\phi(t)}_{0}{\frac{1}{2}\frac{\partial B R^2}{\partial t} d\phi} -  \frac{1}{c} \frac{B R^2}{2} \frac{d\phi}{dt}  = -\frac{B R^2}{2c} \Omega
\end{align}
where we have used Leibniz's rule of for integration with a time changing limit of integration. 

\subsection{The black hole battery} 
Remarkably, it turns out that the Faraday
disc behaves similarly to a BH moving through a magnetic field. The analogy is
spelled out in Part II of this thesis, but if we take for now that the BH
orbiting the NS acts as a conductor with size equal to its event horizon
\citep{MPBook}, then we can calculate the emf generated by the BHNS system.

In the BHNS system, currents are carried by electrons and positrons spiraling
along magnetic field lines, hence the sliding wires of the Faraday disc
example are replaced with B-field lines moving across the BH horizon; the same
lines of magnetic field generate the magnetic flux piercing the moving BH
horizon. Then a closed circuit in the BHNS case traces the $B$-field lines
leaving the NS surface to a time dependent boundary on the BH horizon, crosses
the horizon and trace back along a B-field line to the NS. We take the magnetic field
to be that of a dipole attached to the NS, $|\Bvec| = B_{\rm NS} R^3_{\rm NS}
r^{-3}$, and consider two field lines separated by distance $2R_H$ moving in the
$x$ direction relative to the horizon at speed $v_f$. Then we find a result
for the horizon voltage analogous to the Faraday disc case, and nearly
identical to that presented in Chapters \ref{ch:Rindler} and \ref{ch:NSBH_Fireball},
\begin{align}
\xi_{\rm BH} &= - \frac{1}{c}\frac{d}{dt} \int_{\Sigma(t)}{\Bvec \cdot d\Avec} =  - \frac{\pi R_H}{c} \int^{v_f t}_{0} {B(r) dx}  \\
&=  - \frac{\pi R_H}{c}  \int^{v_f t}_{0}{ \frac{\partial B(a)}{\partial t} dx}  - \pi R_H B(r) \frac{v_f}{c}    \\
& \sim    -R_H \left[r \frac{\left(\Omega_{\rm bin}  - \Omega_{\rm NS}\right)}{c} + \frac{R_H\Omega_{BH}}{c}\right] B_{\rm NS}  \left( \frac{R_{\rm NS}}{r} \right)^3
\end{align}

Here we have assumed that $B(r)$ does not vary across the BH horizon and
evaluate it at the binary separation $r=a$. In the last line we have written
$v_f$ in terms of the binary orbital frequency $\Omega_{\rm bin}$, NS spin
frequency $\Omega_{\rm NS}$, and BH horizon spin frequency $\Omega_{BH}$
\citep[see][]{McL:2011}. This becomes the maximum voltage over one hemisphere
when $R_H$ is the horizon radius. We call the mechanism which generates this voltage in the BHNS system, the BH-battery.
%There an infinite number of these circuits at all times.

In the case of the Faraday disc, the energy which can be harvested
electromagnetically (\textit{e.g.}, by heating the resistor in Figure
\ref{Fig:FDschem}) comes from the energy put into spinning
the disc. In the case of a BHNS binary, the electromagnetic potential energy
of the induced horizon voltage comes from the binary orbital energy, and as
detailed in Part II of this thesis, the available electromagnetic energy could
power luminosities observable from galactic distances (kpc) out to cosmic
distances (Gpc) depending on the NS magnetic field strength at merger.


This result, that BHNS binaries could power high luminosity EM counterparts
without disrupting the NS was first put forth by \cite{McL:2011}. This
work was expanded upon by Chapter \ref{ch:Rindler} of this thesis
\cite{DL:2013}, which finds relativistic solutions for the EM fields of
a magnetic dipole, in arbitrary motion outside of a horizon.

Numerical works have recently tackled this problem in general relativistic,
force-free simulations, which solve the Einstein-Maxwell equations in the
limit that $\Evec \cdot \Bvec$ is everywhere zero, and hence there are no
accelerating forces on electrons \citep{Paschalidis:2013}. Both types of
simulations estimate the observable luminosity via a Poynting flux measured at
the outer edge of the simulation domain. The simulation estimates match the
analytic arguments of \cite{McL:2011} and \cite{DL:2013}. However, a true understanding of the
emission from BHNS systems requires more than this; it requires a radiation
mechanism, something to stick into the BH-battery circuit that will shine.

The emission of EM radiation ultimately must come from dissipation of the BH-
battery power in the joint BHNS magnetosphere. The classic paper by
\cite{GJ:1969} shows that if a spinning NS is immersed in its own magnetic
dipole field, it generates an electric field with components parallel to the
magnetic field. The accelerating $\Evec$ field rips electrons from the NS
crust. The accelerating electrons emit curvature radiation which interacts
with the electromagnetic field to generate electron-positrons pairs that go on
to generate more curvature radiation and a pair cascade ensues. The pairs move
to screen the accelerating electric field, until a force-free condition is
met, and the NS is surrounded by the magnetosphere of \citep{GJ:1969}. Such a
situation halts dissipation of the BH-battery power as long as charges can be
replenished to continue screening accelerating electric fields.

However, this does not stop pulsars from shining. As discussed in
\citep{Sturrock:1971} and \citep{RudSuth:1975}, the force free condition
cannot always be sustained globally in the NS magnetosphere. In regions where
the force free conditions are violated (\textit{e.g.} $|\Evec|^2 > |\Bvec|^2$),
or where the current density depletes the space charge more quickly than it can be
refilled, vacuum gaps must form \citep[\textit{e.g.}][]{DaughertyHarding:1982,
ChengRuderman:1986}. In these gaps, a component of the electric field parallel
to the magnetic field cannot be totally screened, particles are accelerated,
and dissipation allows the release of EM radiation.

We assume that similar mechanisms are at play in the BHNS example. There need
only be gaps in the force free magnetosphere, or magnetic reconnection (though
I am not aware of a process by which this will occur in the BHNS magnetosphere) to
release the BH-battery power. In Chapter \ref{ch:NSBH_Fireball}
\citep{DL:2016}, we envision such a mechanism, which results in a fireball
soon after merger, emitting in the hard X-rays and soft $\gamma$-rays. Recently
a similar fate has been envisioned for the analogous NSNS system
\citep{MetzgerNSNS:2016}. Both of these models may soon be
tested by GW observations of coalescing BHNS and NSNS binaries. From such
observations we could learn about the NS magnetic field strengths at merger,
the NS equation of state, and the dynamics of high energy EM fields in curved
spacetime. For now, work can be focused on further understanding dissipation
in the BHNS magnetosphere. Stay Tuned.











\section{Outline of thesis}   

The rest of this thesis is organized as follows. Chapters \ref{ch:CavAcc}
through \ref{ch:Dust} concern MBHBs. Chapter \ref{ch:CavAcc} presents
hydrodynamical simulations for idealized accretion flows around MBHBs on
circular orbits. It is shown that the accretion rates into the cavity cleared
by the black holes is traced by accretion streams which can feed the black
holes at a rate comparable to that of a single black hole. Furthermore it is
shown that, for non-extreme mass ratio binaries, the accretion rates are
strongly modulated on timescales which depend on the binary mass ratio.
Chapter \ref{ch:CBDTrans} further explores the transition between strongly
modulated accretion flows and steady flows finding dynamical evidence for a
transition in CBDs at a binary mass ratio of 1:25. Chapter \ref{ch:CBDTrans}
also explores the dependence of this transition on disc pressure and
viscosity. Chapter \ref{ch:PG1302_a} utilizes the mass ratio dependent theory
of accretion rate variability worked out in Chapters \ref{ch:CavAcc} and
\ref{ch:CBDTrans} to interpret the MBHB candidate PG 1302-102. Chapter
\ref{ch:PG1302_b} extends this interpretation of PG 1302 in the specific case
that PG 1302 is a binary with mass ratio below the circumbinary disc transition of Chapter
\ref{ch:CBDTrans}. In this case, a compelling interpretation for the periodic
light curve of PG 1302 is found in the relativistic Doppler Boost model.
Chapter \ref{ch:Dust} places the Doppler boost model in the larger setting of
AGN, developing a toy model for the reverberation of optical and UV light by a
surrounding dust torus. 
%This model is fit to the IR light curve of PG 1302,finding agreement.

Chapters \ref{ch:Rindler} and \ref{ch:NSBH_Fireball} concern the interaction
of NS magnetic fields and a BH horizon. Chapter \ref{ch:Rindler}  presents
exact relativistic solutions for the vacuum electromagnetic fields of a
magnetic-dipole source in arbitrary motion near an event horizon. The
solutions are used to interpret and elucidate the electromagnetic circuit
which may be hooked up to create high energy EM emission in a BHNS binary.
Chapter \ref{ch:NSBH_Fireball} examines the nature of this high energy EM
emission by hooking up a circuit of Chapter \ref{ch:Rindler} to a metaphorical
light bulb which manifests in the form of a pair fireball brought on by high
energy curvature radiation.



















%------------------------------------------------------------%
%-------------------------OLD STUFF--------------------------%
%------------------------------------------------------------%
%At 09:50 UTC on September 14, 2015, the laser interferometer
%gravitational wave observatory (LIGO) detected gravitational waves from the
%merger of two $\sim 30 \msun$ black holes. This first direct detection of
%gravitational radiation (gravitational waves or GWs for short) and first
%unequivocal confirmation of the existence of black holes has ushered in a new
%era of gravitational wave astronomy. In this era, multi-wavelength and multi-
%messenger (photons, gravitons, neutrinos) observations of GW sources will be
%crucial for making the next steps to...


%, \textit{e.g.}, the vicinity of merging black holes and neutron stars.
%, including the formation of close binaries on multiple scales,
% such as accretion and electromagnetic field dynamics, that occur




%telling us that we would need to put a detector near the event horizon of a matter distribution with typical velocities near the speed of light in order to experience of order unity perturbations to the spacetime metric. Since these conditions are only found around black holes, we can also conclude that, in building a gravitational wave emitter, black holes are the golden standard in components. As we have not yet built black holes in a laboratory, we look to astrophysical sources. The best known astrophysical sources which approach the dimensions discussed above are The mergers of two (or more) compact objects, namely black holes, neutron stars and white dwarfs \citep{}, Cosmic Inflation \citep{}, Cosmological defects such as cosmic strings \citep{}, Neutron star mountains \citep{}, and core-collapse supernovae \citep{}.
%
%This also tells us that the largest possible gravitational waves are generated by mass distributions with velocities approaching the speed of light and packed into a space the size of a black hole event horizon. GWs with larger amplitudes are shielded by an event horizon. Hence, in building a gravitational wave emitter, black holes are the golden standard in components. As we have not yet built black holes in a laboratory, we look to astrophysical sources. The best known astrophysical sources which approach the dimensions discussed above are
%\begin{itemize}
%\item The mergers of two (or more) compact objects, namely black holes, neutron stars and white dwarfs \citep{}. At the time of writing this class of sources is the only to have been detected in gravitational waves\citep{GW091415}. Two such binary systems are the subjects of this thesis.
%\item Inflation \citep[\textit{e.g.}][]{Guzzatti:2016}.
%\item Cosmological defects such as comsic strings  \citep[\textit{e.g.}][]{}.
%\item Neutron star mountains  \citep[\textit{e.g.}][]{}.
%\item Supernovae  \citep[\textit{e.g.}][]{}.
%\end{itemize}

%A final type of GW detector aims to measure very low frequency ($\sim
%10^{-16}$ Hz) gravitational radiation through its imprint on the polarization
%of the cosmic microwave background (CMB). Though we will not discuss it here,
%the search for such polarization of the CMB is actively being pursued out by
%various competing groups \citep{}.







%The broad motivation of this thesis is to examine astrophysical sources of
%gravitational radiation in the context of their astrophysical environments
%and, in this way, deduce the possibilities for EM observations of GW events
%which could produce multi-messenger events or serve as pure EM observations
%identifiers of GW events in absence of a GW detection.

%Removing the simplification of pure vacuum General Relativity, we placing the
%most promising sources of GW radiation into their expected astrophysical
%environments, and what EM signatures do gravitational wave sources generate
%and what can they tell us? A goal of this thesis is indeed to elucidate this
%question for two specific cases of neutron star black hole (BHNS) binaries
%and massive black hole binaries (MBHBs). In order to place this goal in
%greater context, we first survey the known literature on EM signatures of GW
%sources, specifically the inspiral and coalescence of compact objects.

%For brevity we include only those related to mergers of compact objects. In
%any event, it is useful to consider two categories

%The term EM signature of a GW source is a general term which includes
%observable EM emission that is coincident with a GW detection, and also
%events which occur well before or after the GWs, but can still provide unique
%evidence for the system. The former we refer to as EM counterparts, while the
%latter we call EM tracers. Splitting into these two categories:


%\paragraph{Some EM counterparts of compact object mergers} When an EM signal
%is observable in close temporal proximity to an observable GW event, we call
%this an EM counterpart because it is a counterpart to the GWs (I suppose we
%could just as likely call this a GW counterpart to an EM source). Examples in
%the literature include: %\begin{itemize} %\item gas squeezing X-ray flare at
%merger (Menou Cheng, snowplough papers?) %\item disc response to BH recoil
%\item SGRBs %\item ... %\end{itemize} %as well as the BH battery mechanism
%detailed in \ref{} of this thesis.

%\paragraph{Some EM counterparts of compact object mergers} In the case that
%\EM radiation can identify the a GW source




%Perhaps most relavant to the work here, is the ability of joint EM and GW detections to constrain models of EM emission in extreme merger environments. Models for gas accretion and magnetospheres in curved space time are required to predict such EM counterparts to compact object mergers, hence GW measurements which yield binary parameters and distances can set the basic information which goes into EM emission models, allowing us to rule out models based on the observed EM radiation. Models for the generation of short gamma ray bursts (sGRBs) and killanovae \citep{phinney2009mentionsthis}, as well as the models of Part II, could soon be vetted in this way with LIGO detections of BHNS and NSNS inspiral and merger.


%What telescopes?
%Radio ? 
%SKA

%IR
%WISE
%Spitzer...

%Optical surveys
%LSST
%PTF and ZTF
%Catalina

%UV
%HST
%Galex
 
 %X-ray  -  for Fe K-alpha line  and high energy spectrum
 %current - XMM Chandra
 %future: eROSITA? Athena?
 
 %Gamma-Ray
 %Fermi 




% These analytic treatments gave rise to the first numerical studies of fluid
% flow around a binary in the context of discs around binary stars and
% eventually MBHBs... 1D arguments \citep{SyerClarke, Ivanov?} and early 2D SPH
% simulations \citep{Artymowicz:1991} showed that the outward torques of the
% propeller like binary clear a cavity around near equal mass binaries (those
% expected to form hard binaries...) of size approximately twice the binary
% separation, leaving the binary to inspiral in a environment nearly devoid of
% gas. However





%If an EM event can uniquely identify a source of GWs before or after GW emission we obtain a glimpse into the broader evolution of the binary system. 



%But, before the () payout of detecting EM signatures of GW sources, we must
%first predict what the EM counterparts will look like so we know when to look
%and in what frequencies. A goal of this thesis is indeed to
%elucidate this question for two specific cases of neutron star black hole
%(BHNS) binaries and massive black hole binaries (MBHBs) which we turn to now. 

% observations
% 	dual AGN
% 	spin flips
% 	jet wiggles

% Other Observations?\\
% 		cores vs. cusps
% 		spectroscopic techniques - broad lines, spectral notches or excesses
	
% 	Outstanding problems in this story:
% 		do sub-pc BHs exist, do they merger rapidly or do they stall?
	


% I want to note here 
% (0) intrinsic quasar variability
% (1) that the above searches might exclude binaries with multiple freqeuncies due to naure of wavelet analysis
% (2) Calculation of whether these 144 coudl be Doppler beamed plus figure.
% (3) MBHBs candidates claimed since with plusses or minuses
%  the search continues: \citep{Tamara:2015}
% 	spectroscopic methods:
% 	 BLRs:
% 		\citep{DecarliDott:2013:SpecMBHBcandI}
% 		 Halpern's group?
%      CBD cutoffs:

%      EM time domain
%       Graham a, b
%       Charisi+
%       Zheng, Z.

% (4) IR STUFF JUN ET AL AND CHAPTER 5!



%picture has been altered in the last decade...



% The advent of numerical simulations which capture the 2D, non-axisymmetric
% nature of the binary disc interaction also found that accretion onto the
% binary is not halted by a the torque barrier of the binary, rather accretion
% rates can even be increased from the point mass steady accretion case because
% the binary can vioently pull streams of gass form a circumbinary cavity edge.
% The gas from these streams forms mini-discs around each binary component....
% explain accretion process with figure with vectors...

%Chapter 2 provides evidence for further birfurcations
%for near equal mass ratio systems and Chapter 3 elaborates on the physical
%mechanisms driving the more subtle of these transistion.








% %%% 1D Summary
% Early papers: 
% 	Type I: 
% 		Lynden-Bell and Kalnajs 1972 - resodnances and spiral arms in galactic setting
% 		Lin and Papaloizou 1979 a - angular momentum transport:Type I migration - tidal truncation of CBDs - application to dwarf novae
% 		GT 1980 - first to suggest migration - from resonant tides
% 		Meyer Sicardy - MVSic87


% 	Type II
% 		Syer Clarke disc vs sec dominated - app to MBHBs
		
% 		b - Tidal truncation of CBDs applied to contact binaries and the formation of commensurable satellites in the solar system.
% 		Lin and Papaloizou 1986 a - full nonlinear interaction f disc and sattelite, self grav, range of c_s and nu - density wave propagation studied- applied to protoplanet and the primordial solar nebula
% 		b - dynamical evolution of the disc and the orbital migration of the protoplanet in a self-consistent manner is considered - Type II viscous evolution rate derived here - numerical stuff?
% 		Lin Papa 1993

% 	Gap Clearing 
% 		The above Lin Papa papers
% 		Arty and Lubow 1994, 1996
	
% 	Both:
% 	Hourigan Ward 1984
% 	Ward 1997

% 	Eccentricity excitation/damping
% 	Ward 1988
% 	Goldreich Sari 2003

% Present day answers to earlier work:
% Type I:
% Rafikov and Dong etc
% Duffell

% Type II:
% Edgar 2008
% FTV Duffell 2014
% Durman and Kley? 2015

%Syer and Clarke: secondary BH can clear a gap in the CBD, and bank up material behind it, causing a change in the spectarl and possibly continuum emission from the AGN. They also introduced the idea od Type II secondary dominated vs. disc dominated mogration.

% Retrograde and warps alignment
% Nixon+2011 - retro
% Nixon+2012 - alignment to retro of misaligned
% Hayasaki+2013 - warped/alignment
% Nixon+2013 - tearing up the disc - misaligned
% ReodigSesana:2014 - retro vs pro
% DoganNixonKingPrice:2015 - tearing up misaligned
% BankertShiKrolik:2015 - retro
% Amaro-Seoane+2016 - retro

% EMPHASIZE THE QUESTIONS IN THIS THESIS: Can accretion make EM signature of MBHB and can we use it to identify a pop of MBHB candidates?













%Palenzuela:2013


% These works have understood the energetics and mechanisms for generating
% electric potential in the inspiralling BHNS system, however they do not fully
% adress how this potential energy is relased as observable EM radiation.




% This analytic work has been tested by numerical simulations of the 

% Numerical simulations:
% Shapiro, Palenzuela for NS-NS: Full resisitive,
% general relativistic, MHD simulatiosn have also been carried out for merging
% NSNS binaries, where a similar unipolar inductor mechansim is at play
% \citep{Palenzuela..}.
% other theory:
% Lai
% Piro


% not jsut BHs, but planets WDs, any situtation wher a conducting surface moves through an astrophyiscal magnetic field.




% previous work on this...


% %magnetosphere phsyics
% While the BH batterey provides a power source, the mechanisms by which this power is dissipted is crucial for predicting observational consequences. 

% Dissipation in NS magnetospheres:
% GJ69, RS72, and Force Free?\\
% How does disipation ensue? - Reconnection, Gaps?

% Conclusions:
% %Problems to surrmount?
% How do we form close BHNSs?
% what are the NS field strengths at merger?\\

% %LIGO relation
% Review of Pop Synth for BHNS LIGO?\\
% prospects in the LIGO era


