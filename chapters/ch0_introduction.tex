\chapter[Introduction]{Introduction} \label{ch:intro}

%?There are those who love to get dirty and fix things. They drink coffee at dawn, beer after work. And those who stay clean, just appreciate things. At breakfast they have milk and juice at night. There are those who do both, they drink tea.? 
%? Gary Snyder

%?Nature is not a place to visit. It is home.? 
%? Gary Snyder


%?Three-fourths of philosophy and literature is the talk of people trying to convince themselves that they really like the cage they were tricked into entering.? 
%? Gary Snyder


%?With no surroundings there can be no path, and with no path one cannot become free.? 
%? Gary Snyder, Practice of the Wild

\vspace{-16pt} \begin{chapquote}{Gary Snyder quoting Ezra Pound para-phrasing Lu Ji} \singlespacing ``When making an axe handle, the pattern is near at hand.'' 
 \end{chapquote} \vspace{-8pt}
\noindent\makebox[\linewidth]{\rule{0.5\textwidth}{0.5pt}} \vspace{1pt}


%1)  Introduce the concept of gravitational radiation and its importance.
%	a) what are GWs
%	b) What are the expected sources of GWs
%	c) How are GWs detected - list of experiments and their sensitivity (Figure?)
%2) Introduce the role EM counterparts can play 
%	a) with GW detection
%	b) without GW detection


Scientific discovery is aided and driven by observations. Before 2015, all
such observations and corresponding scientific conclusions were founded on the
detection of photons \footnote{plus a few nuetrinos \citep{}}, the messenger
of the electromagnetic (EM) interaction. At 09:50 UTC on September 14, 2015
the laser interferometer gravitational wave observatory (LIGO) observed the
universe for the first time in gravitons, or gravity waves (GWs), the
messenger of the gravitational interaction. Even before the detection of GWs,
the importance of combining information from EM and gravitational views was
recognized \citep[\textit{See e.g.}][for some early discussions]{first EM
mentions}. The work laid out in this thesis is a contribution to this effort,
to maximize our observations of GW sources by predicting the nature of EM
signatures that should accompany them, or signify their existence beforehand.
Such an endeavor not only provides ways to find sources of GWs and learn about
their operation, but also drives investigation into the astrophysics that
creates GW sources, and into the workings of physical processes in the extreme
environments that generate gravitational radiation. We proceed by briefly
discussing the expected sources of GWs, their detection, and the utility of
their possible EM signatures. We then introduce two specific GW sources that
are the topic of this thesis.

%, \textit{e.g.}, the vicinity of merging black holes and neutron stars.
%, including the formation of close binaries on multiple scales,
% such as accretion and electromagnetic field dynamics, that occur


\subsection{The gravitational side} %Gravitational radiation is generated when
the shape of a mass-energy distribution changes in time. More precisely, when
quadrapole and higher moments of a mass distribution have a changing current,
metric perturbations Gravitational radiation is generated by the acceleration
of quadrapole or higher moments of a mass-energy distribution. The result is
the generation of metric perturbations \begin{equation} h_{i j} = \frac{2G}{d
c^4} \ddot{Q}_{ij}, \end{equation} that propagate through spacetime as
gravitational waves, carrying the information of a changing gravitational
field. Here $G$ and $c$ are the usual gravitational constant and the speed of
light, while $d$ is the distance from observer to source of radiation, $
\ddot{Q}_{ij}$ is the second time derivative of the mass-energy quadrapole
tensor and $h_{ij}$ is the dimensionless strain tensor which measures
fractional changes in proper distances. The units of $\ddot{Q}_{ij}$ are a
mass times a velocity squared. Hence the wave amplitude is set by twice the
kinetic energy put into accelerating the quadrapolar moment of a mass-energy
distribution%\footnote{\textit{e.g.} spinning a rod of length $r$ and mass $M$
at angular frequency $\omega$ has rotational energy $1/2 M r^2 \omega^2$} ,
times a coupling constant $2 Gc^{-4}d^{-1}$. The coupling constant is
determined by the strength of the quadrapolar tidal field; the minuscule size
of this coupling constant is perhaps the reason why it has taken a century
since their prediction to detect gravitational waves. For example, the
gravitational wave strain from two point masses of total mass M, on a circular
orbit of separation $a$ is of order \begin{equation} h \sim \frac{2G}{d c^4} M
v^2 \sim \frac{G M }{a c^2} \frac{G M }{d c^2}. \end{equation} Even for a
binary consisting of two suns, orbiting as rapidly as possible, $a=R_{\odot}$,
and within in our galaxy $d \sim 1$ kpc the strain is incredibly small:
$h\sim10^{-22}$.

To experience gravitational wave strains of order unity, one must put a
detector at a distance $d= 2GMc^{-2} (v/c)^2$ from a system of mass $M$ and
with typical velocities $v$. This distance is of order the gravitational
radius, or the event horizon scale of a black hole. As we have not yet built
black holes in a laboratory, we look to astrophysical sources. The best known
astrophysical sources which approach the above dimensions, and could occur
close enough and frequently enough to be detectable, are the mergers of two
(or more) compact objects, namely black holes (BHs), neutron stars (NSs), and
white dwarfs (WDs) \citep{}, cosmic Inflation \citep{}, cosmological defects
such as cosmic strings \citep{}, Neutron star mountains \citep{}, and core-
collapse supernovae \citep{}. For the remainder of this thesis we focus on the
first example, and specifically the mergers BHs and NSs binaries.



%telling us that we would need to put a detector near the event horizon of a matter distribution with typical velocities near the speed of light in order to experience of order unity perturbations to the spacetime metric. Since these conditions are only found around black holes, we can also conclude that, in building a gravitational wave emitter, black holes are the golden standard in components. As we have not yet built black holes in a laboratory, we look to astrophysical sources. The best known astrophysical sources which approach the dimensions discussed above are The mergers of two (or more) compact objects, namely black holes, neutron stars and white dwarfs \citep{}, Cosmic Inflation \citep{}, Cosmological defects such as cosmic strings \citep{}, Neutron star mountains \citep{}, and core-collapse supernovae \citep{}.
%
%This also tells us that the largest possible gravitational waves are generated by mass distributions with velocities approaching the speed of light and packed into a space the size of a black hole event horizon. GWs with larger amplitudes are shielded by an event horizon. Hence, in building a gravitational wave emitter, black holes are the golden standard in components. As we have not yet built black holes in a laboratory, we look to astrophysical sources. The best known astrophysical sources which approach the dimensions discussed above are
%\begin{itemize}
%\item The mergers of two (or more) compact objects, namely black holes, neutron stars and white dwarfs \citep{}. At the time of writing this class of sources is the only to have been detected in gravitational waves\citep{GW091415}. Two such binary systems are the subjects of this thesis.
%\item Inflation \citep[\textit{e.g.}][]{Guzzatti:2016}.
%\item Cosmological defects such as comsic strings  \citep[\textit{e.g.}][]{}.
%\item Neutron star mountains  \citep[\textit{e.g.}][]{}.
%\item Supernovae  \citep[\textit{e.g.}][]{}.
%\end{itemize}


The are multiple methods for detecting gravitational waves from merging
compact objects. Just as for EM radiation, detector design depends on the
radiation frequency. The gravitational wave frequency for a binary on a
circular orbit is given by twice the orbital frequency\footnote{Eccentric
orbits emit gravitational waves over a spectrum of frequencies spanning the
circular frequency and its higher order harmonics \citep{}.} \begin{equation}
f_{\GW} = 2 f_{\orb} \approx  \frac{1}{\pi} \sqrt{\frac{G M }{a^3}} =
\frac{1}{\pi} t^{-1}_{\rm{G}} \left( \frac{a}{r_{\rm{G}}} \right)^{-3/2}
\end{equation} where $M$ is the total binary mass and $a$ is the binary
separation and $t_{\rm{G}} \equiv GMc^{-3}$ is the gravitational time while
$r_{\rm{G}} \equiv GMc^{-2}$ is the gravitational radius.  For astrophysical
black holes, which range in mass from $\sim 1 \Msun \rightarrow 10^{10}
\Msun$, the gravitational wave frequency covers ten orders of magnitude.
Assuming $a = 2GM/c^2$ at merger, this range gives $f_{\GW} = 10^4 \rightarrow
10^{-6}$ Hz. Considering also GW emission during the inspiral stage, the
largest black holes emit at frequencies of $f_{\GW} \sim 10^{-9}$ Hz at
separations of order $100 GM/c^2$.


This wide range of astrophysicaly interesting frequencies is currently covered
by three different detector designs. From high to low frequencies, the first
two use laser interferometers to detect the very small distance change between
two test masses when a GW passes through them. The laser interferometer
gravitational wave observatory (LIGO) is sensitive to GW frequencies ranging
from $\sim 10 \rightarrow 10^4$ Hz with a peak strain sensitivity at
$\sim10^{2}$ Hz of $h \gsim 10^{-22}$. This makes LIGO sensitive to the
inspiral, merger, and ringdown of stellar mass compact object binaries
consisting of BHs and NSs. LIGO could also detect GWs from the mountains on
millisecond pulsars \citep{}, or the stellar oscillations due to giant core
collapse supernovae \citep{}. LIGO's localization capabilities are limited to
a rather broad $\sim$few square degrees, but will increase when the two
existing interferometers are joined by their international counterparts: VIRGO
\citep{} in Italy, GEO600 in Germany \citep{}, KAGRA being built in Japan
\citep{}, and in LIGO-India approved in February of 2016 \citep{}.

At frequencies below $\sim 1$ Hz, earth related vibrations swamp the LIGO
sensitivity making detection of sub Hz sources impossible \citep{}. For this
reason, the Laser Interferometer Space Antenna (LISA) was envisioned
\citep{LISAbeginnings}.  As a kind of LIGO in space, LISA-like detectors are
planned to be sensitive to lower frequency GWs in the range $10^{-5}
\rightarrow 1$ Hz with a peak sensitivity of $h \gsim 10^{-23}$ over a range
of $0.01 \rightarrow 0.1$ Hz. LISA will be oriented in a orbit around the Sun
such that its changing orientation in time will allow localization of sources
to with 10 square degrees \citep{}. LISA sources include the inspiral and
merger of $10^4 \rightarrow 10^7 \Msun/(1+z)$ MBHBs in galactic nuclei at
redshift $z$ \citep{}, the orbits of thousands of galactic binaries \citep{},
and the inspiral of NS and stellar BH binaries before they reach the LIGO band
\citep{Sesana:2016:LIGOLISA and refs therein}.


A second type of gravitational wave detector looks to nature's clocks, the
pulsars, to act as a galactic timing array. The so-called Pulsar Timing Arrays
(PTAs) search for deviations in the arrival time of the pulses from
millisecond pulsars. Timing deviations on the order of nano-seconds,
correlated over multiple pulsars in our galaxy would signify the presence of
very long wavelength gravitational waves, with frequencies ranging from
$\sim10^{-9} \rightarrow 10^{-6}$ Hz (wavelengths of parsecs to mill-
parsecs!). Such low frequency radiation is expected from the inspiral of the
largest BHs in the universe in galactic centers. For the closest ($z\lsim1$)
MBHB inspirals, the PTAs could pick out the GW signal from an individual
event, otherwise the PTAs will measure a stochastic background of GWs from
MBHBs spiraling together throughout the universe. The magnitude and frequency
dependence of the GW background holds information on the role of gas and stars
in driving the binary inspiral through the PTA band and is an important probe
of the MBHB population \citep[\textit{e.g.}][]{}. The PTAs may also be
sensitive to more exotic sources of gravitational radiation including the
interactions of cosmic strings. Localization of individual GW sources by the
PTAs will be poorly constrained to $\sim 100$ square degrees \citep{}.
Currently there are three active groups monitoring pulsars for use as a GW
detector, the Parkes Pulsar Timing Array \citep[PPTA][]{PPTA}, the European
Pulsar Timing Array \citep[EPTA][]{EPTA}, and the North American Nanohertz
Observatory for Gravitational Waves \citep[NANOGrav][]{NANOGrav}. The
International Pulsar Timing Array \citep[iPTA][]{iPTA} is a consortium between
these groups.





%A final type of GW detector aims to measure very low frequency ($\sim 10^{-16}$ Hz) gravitational radiation through its imprint on the polarization of the cosmic microwave background (CMB). Though we will not discuss it here, the search for such polarization of the CMB is actively being pursued out by various competing groups \citep{}. 








\subsection{The electromagnetic side} When black holes interact with gas and
strong electromagnetic fields, they are sources of bright EM radiation on
their own (\emph{e.g.} active galactic nuclei and x-ray binaries). Boasting
surface fields of $\sim 10^{12}$ G and up, neutron stars carry with them an
enormous supply of potential EM energy and are themselves observable within
our Galaxy. It is thus plausible that, in pairs, BHs and NSs could generate
bright EM emission. Due to the potential modulation of an EM signal from
binary orbital motion, as well as extreme energies that can be experienced at
the end of the binary death spiral, such a signature may not only be bright,
but uniquely identifiable as well.

If an electromagnetic signature can be identified with a GW event, we call it
an EM counterpart. Such an EM counterpart will allow localization of the GW
source on the sky, which as we saw above, is not easy to do with GWs alone.
Locating a bird by listening to its song vs. sighting it with your eyes, comes
close to the analogous problem of identifying a source location with multiple
gravitational wave detectors vs. pin-pointing its location with a telescope.
For interferometer detectors (or just LISA), sky localization improves the
precision of the distance measurement, as this is largely limited by pointing
error \citep{HughesLDist:2002}. Localization will also give us contextual
clues to the nature of the source: is it in an galactic nucleus, at the heart
of a globular cluster, or in the outskirts of a galaxy cluster? Such
information could constrain formation scenarios.

If the EM observation yields not just a sky localization but also a redshift,
the corroboration of a precise GW measured distance and an electromagnetically
measured redshift can yield a precise measurement of the Hubble constant and
other cosmological parameters \citep{Schutz:1986, KrolakSchutz:1987
,ChernoffFinn:1993, HolzHughes:2005, Dalal:2006, KocsisFrei+2006,
Nissanke:2010} as well as constrain fundamental physics such as the nature of
gravity on large scales \citep{DeffayetMenou:2007}.

If an EM observation can make independent measures of the binary parameters,
removing degeneracies in determining binary parameters \citep{HughesHolz 2003}

reduce S/N for GW detection \citep{KochanekPiran:1993, HarryFairhurst 2011}


unravel astro context (Phinney 2009, Mandel and O'Shaughnessy 2010 e.g.
Fong+2010)




High precision measurements of masses and spins vectors, binary orbital
parameters, Lum det to high precision (Lang Hughes 2006, Vecchio 2004)

With a counterpart, can measure Eddington ratios and other accretion physics
related stuff (like my predictions not known at the time of writing this
paper) - Kocsis, Frei, Haiman, Menou 2006, Kocsis, Haiman , Menou 2008)



Even if an EM signature can identify a source of GWs, before, or after the GWs
are detectable, we can still glean valuable information. We can identify a
population of such sources...can still...identify, survey population in a
state before or after merger...


but first we must predict what the EM counterparts will look like so we know
when and to look and in what frequencies. A goal of this thesis is indeed to
elucidate this question for two specific cases of neutron star black hole
(NSBH) binaries and massive black hole binaries (MBHBs) which we turn to now.

%Some examples

%Some uses

%The broad motivation of this thesis is to examine astrophysical sources of
%gravitational radiation in the context of their astrophysical environments
%and, in this way, deduce the possibilities for EM observations of GW events
%which could produce multi-messenger events or serve as pure EM observations
%identifiers of GW events in absence of a GW detection.

%Removing the simplification of pure vacuum General Relativity, we placing the
%most promising sources of GW radiation into their expected astrophysical
%environments, and what EM signatures do gravitational wave sources generate
%and what can they tell us? A goal of this thesis is indeed to elucidate this
%question for two specific cases of neutron star black hole (NSBH) binaries
%and massive black hole binaries (MBHBs). In order to place this goal in
%greater context, we first survey the known literature on EM signatures of GW
%sources, specifically the inspiral and coalescence of compact objects.

%For brevity we include only those related to mergers of compact objects. In
%any event, it is useful to consider two categories

%The term EM signature of a GW source is a general term which includes
%observable EM emission that is coincident with a GW detection, and also
%events which occur well before or after the GWs, but can still provide unique
%evidence for the system. The former we refer to as EM counterparts, while the
%latter we call EM tracers. Splitting into these two categories:


%\paragraph{Some EM counterparts of compact object mergers} When an EM signal
%is observable in close temporal proximity to an observable GW event, we call
%this an EM counterpart because it is a counterpart to the GWs (I suppose we
%could just as likely call this a GW counterpart to an EM source). Examples in
%the literature include: %\begin{itemize} %\item gas squeezing X-ray flare at
%merger (Menou Cheng, snowplough papers?) %\item Disk response to BH recoil
%\item SGRBs %\item ... %\end{itemize} %as well as the BH battery mechanism
%detailed in \ref{} of this thesis.

%\paragraph{Some EM counterparts of compact object mergers} In the case that
%\EM radiation can identify the a GW source

 
 

 



The specific sources of GWs studied in this thesis are the inspiral and merger
of MBHBs in galactic nuclei and the merger of magnetized NSs with $\gsim 10
\Msun$ BHs.

\section{Part I: Massive Black Hole Binaries}
% This section should set up - what MBHBs are and how we think they form, what EM signatures do they have, what observations. Then should go into specifics of what this work will focus on, how it fits in the big picture and in the end what we can learn from MBHBs
\subsection{Formation}
Origin story + Final Parsec Problem\\
How do gas and BHs get to the nucleus\\
do sub-pc BHs exist, do they merger rapidly or do they stall?
Observations and theory to support

\subsection{Accretion as a periodic EM signature}
The work in Part I of this thesis focuses on the sub-pc regime where the binary could interact for long periods of time with a circumbinary gas disk...
List of past work\\

Theory of disk+binary interaction\\
The theoertical framework for this problem goes back to the planetary literature on sateelite and disk interactions in proto-planetery disks and the rings of saturn...
also later analytic MBHB work: Shapiro, Seyr Clarke, Ivanov

These analytic treatments then gave rise to the first numerical studies of fluid flow around  abinary in the context of disks around binary stars. In the late 2000's these studies re-invented themselves in application to the MBHBs. spell out scientific findings of each (Hayasaki, MM08, Cuadra, Noble, Roedig, ShiKrolik, KingNixon, D'Orazio, Farris, ClarkeYoung(?), Munoz, Lodato)
GR regime: Farris and Palenzuela.
Simulations -  - idealized CBD setup sims\\


observations:\\
dual AGN
spin flips
jet wiggles
spectroscopic techniques
The first EM time domain observations- Graham, D'Orazio, Charisi 
(FIGURE - Doppler candidates)



\section{Part II: Stellar Black Hole + Neutron Star Binaries}
The merger of NSs and stellar BHs will generate GWs detectable by the Laser
Interferometer Gravitational-Wave Observatory (LIGO). Binaries with BHs will
generate the highest amplitude GW signals, but a binary containing a NS could
produce a bright EM signal, making BHNS systems especially interesting sources
of EM+GW emission.

The tidal disruption of a NS by its BH partner could generate a $\gamma$-ray
burst after merger \citep{NPP:NSBH_GRB:1992}. However, it is under-appreciated
that most BHs should be large enough ($> 6M_\odot $) to swallow their NSs
whole, causing the mergers of most BHNS binaries to be dark. 
%Go into some details of when disruption occurs here! 
Although the distribution of BH masses
which will end their lives merging with a NS is unknown, it is interesting to
note that the BH mass distribution inferred from BHs in X-ray binaries peaks
around $8 \Msun$ \citep{Orzel:2008} and the only known BH binary contained BH
masses of $\sim 30 \Msun$, which would certainly swallow a NS hole. However,
each of these formation channels may be independent and not applicable to a
NSBH system. As additional motivation, LIGO is the most sensitive at a
frequency of $\sim 200$ Hz, this is the gravitational wave frequency at
coalescence for a NS of mass $1.4 M_{\Msun}$ in a circular orbit with a BH of
mass few $\sim100 \Msun$. If such binaries occur in nature, they have the
potential to be high signal to noise LIGO detections, and will certainly not
disrupt the NS. The above motivates an exploration of EM counterparts to non-
disrupting NSBH systems.



%Explanation of BH battery - Review of UI models in astro\\


%magnetosphere phsyics
Review of GJ69, RS72, and Force Free?\\


%Problems to surrmount?

%LIGO relation
Review of Pop Synth for NSBH LIGO?\\
prospects in the LIGO era









\section{Outline of thesis} 
The rest of this thesis is organized as follows.
Chapter 2 presents hydrodynamical simulations for idealized accretion flows
around MBHBs on circular orbits. It is shown that the accretion rates into the
cavity cleared by the black holes is traced by accretion streams which can
feed the black holes at a rate comparable to that of a single black hole.
Furthermore it is shown that, for non-extreme mass ratio binaries, the
accretion rates are strongly modulated on timescales which depend on the
binary mass ratio. Chapter 2 further explores the transition between strongly
modulated accretion flows and steady flows finding dynamical evidence for a
transition in CBDs for a binary mass ratio of 1:25. Chapter 2 also explores
the dependence of this transition on disk pressure and viscosity. Chapter 3
utilizes the mass ratio dependent theory of accretion rate variability worked
out in Chapters 2 and 3 to interpret the MBHB candidate PG 1302-102. Chapter 4
extends this interpretation of PG 1302 in the specific case that PG 1302 is a
binary with mass ratio below the CBD transition of Chapter 3. In this case, a
compelling interpretation for the periodic light curve of PG 1302 is found in
the relativistic Doppler Boost model. Chapter 5 places the Doppler boost model
in the larger setting of AGN, developing a toy model for the reverberation of
optical and UV light by a surrounding dust torus. This model is fit to the IR
light curve of PG 1302, finding agreement.

Chapter 6 presents relativistic solutions for the electromagnetic fields of a
magnetic dipole source in arbitrary motion near an event horizon. The
solutions are used to interpret and elucidate the more complicated



















%------------------------------------------------------------%
%-------------------------OLD STUFF--------------------------%
%------------------------------------------------------------%
%At 09:50 UTC on September 14, 2015, the laser interferometer
%gravitational wave observatory (LIGO) detected gravitational waves from the
%merger of two $\sim 30 \msun$ black holes. This first direct detection of
%gravitational radiation (gravitational waves or GWs for short) and first
%unequivocal confirmation of the existence of black holes has ushered in a new
%era of gravitational wave astronomy. In this era, multi-wavelength and multi-
%messenger (photons, gravitons, neutrinos) observations of GW sources will be
%crucial for making the next steps to...

%What telescopes?
%Radio ? 
%SKA

%IR
%WISE
%Spitzer...

%Optical surveys
%LSST
%PTF and ZTF
%Catalina

%UV
%HST
%Galex
 
 %X-ray  -  for Fe K-alpha line  and high energy spectrum
 %current - XMM Chandra
 %future: eROSITA? Athena?
 
 %Gamma-Ray
 %Fermi 