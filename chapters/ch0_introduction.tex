\chapter[Introduction]{Introduction} \label{ch:intro}

%?There are those who love to get dirty and fix things. They drink coffee at dawn, beer after work. And those who stay clean, just appreciate things. At breakfast they have milk and juice at night. There are those who do both, they drink tea.? 
%? Gary Snyder

%?Nature is not a place to visit. It is home.? 
%? Gary Snyder


%?Three-fourths of philosophy and literature is the talk of people trying to convince themselves that they really like the cage they were tricked into entering.? 
%? Gary Snyder


%?With no surroundings there can be no path, and with no path one cannot become free.? 
%? Gary Snyder, Practice of the Wild

\vspace{-16pt} \begin{chapquote}{Gary Snyder quoting Ezra Pound para-phrasing Lu Ji} \singlespacing ``When making an axe handle, the pattern is near at hand.'' 
 \end{chapquote} \vspace{-8pt}
\noindent\makebox[\linewidth]{\rule{0.5\textwidth}{0.5pt}} \vspace{1pt}


%1)  Introduce the concept of gravitational radiation and its importance.
%	a) what are GWs
%	b) What are the expected sources of GWs
%	c) How are GWs detected - list of experiments and their sensitivity (Figure?)
%2) Introduce the role EM counterparts can play 
%	a) with GW detection
%	b) without GW detection
	

%At 09:50 UTC on September 14, 2015, the laser interferometer gravitational wave observatory (LIGO) detected gravitational waves from the merger of two $\sim 30 \msun$ black holes. This first direct detection of gravitational radiation (gravitational waves or GWs for short) and first unequivocal confirmation of the existence of black holes has ushered in a new era of gravitational wave astronomy. In this era, multi-wavelength and multi-messenger (photons, gravitons, neutrinos) observations of GW sources will be crucial for making the next steps to...

Scientific discovery is aided and driven by observations. Before 2015, all such observations and all scientific conclusions on the workings of the universe were founded on the detection of photons \footnote{plus a few nuetrinos \citep{}}, the messenger of the electromagnetic force. At 09:50 UTC on September 14, 2015 the laser interferometer gravitational wave observatory (LIGO) observed the universe for the first time in gravitons, or gravitational radiation, or gravity waves (GWs), the messenger of the gravitational interaction. Even before the detection of GWs (and more so after), the importance of combining information from EM and gravitational views was recognized. The work laid out in this thesis is a contribution to this effort, to maximize our observations of GW sources by predicting the nature of EM signatures that should accompany them, or signify their existence beforehand. As the work in this thesis shows, such an endeavor not only provides ways to find sources of GWs and learn about their operation, but also drives investigation into the physical processes, such as accretion and electromagnetic field dynamics, in the extreme environments that generate gravitational radiation, \textit{e.g.} in the vicinity of merging black holes and neutron stars. We proceed by first briefly discussing the expected sources of GWs, their detection, and their expected EM signatures and utility. We then introduce two specific GW sources that are the topic of this thesis.





\subsection{The gravitational side}
Gravitational radiation is generated when the shape of a mass-energy distribution changes in time. More precisely, when quadrapole and higher moments of a mass distribution have a changing current, metric perturbations
\begin{equation}
h_{i j} = \frac{2G}{d c^4} \ddot{Q}_{ij},
\end{equation}
are generated and propagate through space as gravitational waves. Here $G$ and $c$ are the usual gravitational constant and the speed of light, while $d$ is the distance from observer to source of radiation, and $ \ddot{Q}_{ij}$ is the second time derivative of the mass-energy quadrapole tensor. The units of $\ddot{Q}_{ij}$ are a mass times a velocity squared. Hence the wave amplitude is set by twice the kinetic energy $M v^2$ put into changing the quadrapolar moment of a mass-energy distribution times a coupling constant $2 Gc^{-4}d^{-1}$. The coupling constant is determined by the strength of the quadrapolar tidal field. The minuscule size of this coupling constant is perhaps the reason that it has taken a century since their prediction to detect gravitational waves. That is, to experience gravitational wave amplitudes of order unity, one must put a detector at a distance $d= 2GMc^{-2} (v/c)^2$ from a system of mass $M$ and with typical velocities $v$. This distance is of order the gravitational radius, telling us that the largest possible gravitational waves are generated by mass distributions with velocities approaching the speed of light and packed into a space the size of a black hole event horizon. GWs with larger amplitudes are shielded by an event horizon. Hence, in building a gravitational wave emitter, black holes are the golden standard in components. As we have not yet built black holes in a laboratory we look to astrophysical sources. The best known astrophysical sources which approach the dimensions discussed above are
\begin{itemize}
\item The mergers of two (or more) compact objects, namely black holes, neutron stars and white dwarfs \citep{}. At the time of writing this class of sources is the only to have been detected in gravitational waves\citep{GW091415}. Two such binary systems are the subjects of this thesis.
\item Inflation \citep[\textit{e.g.}][]{Guzzatti:2016}.
\item Cosmological defects such as comic strings  \citep[\textit{e.g.}][]{}.
\item Neutron star mountains  \citep[\textit{e.g.}][]{}.
\item Supernovae  \citep[\textit{e.g.}][]{}.
\end{itemize}


The methods for detecting gravitational waves are nearly as varied as the sources themselves. Just as for EM radiation, detector design depends on the radiation frequency. For the gravitational wave sources most relevant to this thesis, coalescing binary systems, the gravitational wave frequency for a binary on a circular orbit is given by twice the orbital frequency
\begin{equation}
f_{\GW} = 2 f_{\orb} \approx  \frac{1}{\pi} \sqrt{\frac{G M }{a^3}}
\end{equation}
where $M$ is the total binary mass and $a$ is the binary separation \footnote{Eccentric orbits emit gravitational waves over a spectrum of frequencies spanning the circular frequency and its higher order harmonics \citep{}.}.  For astrophysical black holes, which range in mass from $\sim 1 \Msun \rightarrow 10^{10} \Msun$, the gravitational wave frequency covers ten orders of magnitude. Assuming $a = 2GM/c^2$ at merger, this range gives $f_{\GW} = 10^4 \rightarrow 10^{-6}$ Hz. Considering also GW emission during the inspiral stage, the largest black holes emit at frequencies of $f_{\GW} \sim 10^{-9}$ Hz at separations of order $100 GM/c^2$.

Because of this large range of astrophysicaly interesting frequencies, the instruments in operation or design today that aim to detect gravitational waves consist of three types. The first uses laser interferometers to detect the very small () distance change between two test masses when a GW passes through the system. These consist of LIGO \citep{} and the future space based mission eLISA \citep{}. 
LIGO\\
Existing instruments built on Earth to detect high frequency radiation from smaller Mass - LIGO, VIRGO and friends. 
Design sensitivity $h \sim$ at $10 -1000$ Hz
timescale
eLISA\\
The merger of the most massive black holes as well as the inspiral of middle weight MBHBs and nearby white dwarf binaries will be detectable by future space based interferometers such as eLISA.


A second type of gravitational wave detector looks to natures clocks, the pulsars, to act as a galactic timing array. The exquisite timing of the pulse arrival of millisecond pulsars allows the small changes in spacetime... 
PTAs\\
The PTAs consist of three consortiums...
The very long wavelength radiation from the inspiral of the most massive black hole binaries can be detected by the Pulsar timing arrays.


A final type of GW detector aims to measure gravitational radiation through its EM signatures alone...
the imprint of gravitational radiation in the polarization of the CMB. In practice there are many such telescopes trying to do this here are some \citep{} Though will not be the topic of this work.
CMB\\
Gravitational radiation from inflation is thought to permeate the universe at much lower frequencies than the binary populations. Experiments attempting to measure the polarization of the cosmic microwave background is attempting to detect this $\sim 10^{-16}$ Hz relic radiation \citep{}.











\subsection{The electromagnetic side}
%The broad motivation of this thesis is to examine astrophysical sources of gravitational radiation in the context of their astrophysical environments and, in this way, deduce the possibilities for EM observations of GW events which could produce multi-messenger events or serve as pure EM observations identifiers of GW events in absence of a GW detection.
%
What EM signatures do gravitational wave sources generate and what can they tell us? A goal of this thesis is indeed to elucidate this question for two specific cases of neutron star black hole (NSBH) binaries and massive black hole binaries (MBHBs). In order to place this goal in greater context, we first survey the known literature on EM signatures of GW sources. It is useful to separate these into two categories

\paragraph{EM counterparts} When an EM signal is observable in close temporal proximity to an observable GW event, we call this an EM counterpart because it is a counterpart to the GWs (I suppose we could just as likely call this a GW counterpart to an EM source). Examples in the literature include:
\begin{itemize}
\item gas squeezing X-ray flare at merger (Menou Cheng, snowplough papers?)
\item Disk response to BH recoil
\item SGRBs
\item ...
\end{itemize}
as well as the BH battery mechanism detailed in \ref{} of this thesis.

\paragraph{EM only} In the case that EM radiation can identify the a GW source 

 



%This thesis has focused on predicting the observable electromagnetic signatures of two specific sources of gravitational radiation.  So I need to :



The specific sources of GWs studied in this thesis are the inspiral and merger of MBHBs in galactic nuclei and the merger of magnetized NSs with $\gsim 10 \Msun$ BHs.
\section{Part I: Massive Black Hole Binaries}
Origin story + Final Parsec Problem\\

How does gas get to the nucleus\\

List of past work:\\
Theory of disk+binary interaction\\
Simulations\\

The first observations- Graham, D'Orazio, Charisi



\section{Part II: Stellar Black Hole + Neutron Star Binaries.}

Tidal Disruption details - and EM signals from disruption\\

Explanation of BH battery - Review of UI models in astro\\

review of GJ69 and Force Free?\\

Review of Pop Synth for NSBH LIGO\\


prospects in the LIGO era









\section{Outline of thesis}























