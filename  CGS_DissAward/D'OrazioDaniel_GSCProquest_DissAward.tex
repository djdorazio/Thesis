\documentclass[12pt,letterpaper]{article}

%
% Packages to be used in the thesis.
%

\usepackage{amsmath}
\usepackage{amsfonts}
\usepackage{amssymb}
\usepackage{booktabs}
\usepackage{url}

\usepackage[final]{graphicx}

\usepackage{color, hyperref}
\definecolor{linkcolor}{rgb}{0,0,0.2}
\hypersetup{colorlinks=true,linkcolor=linkcolor,citecolor=linkcolor,
            filecolor=linkcolor,urlcolor=linkcolor}
\hypersetup{pageanchor=false}

\usepackage{appendix}
\usepackage{fancyhdr}
\usepackage{url}

\usepackage{subfig}

\usepackage{titlesec}

%DANS ADDITIONS
\usepackage{array}
\usepackage{fixltx2e}
%\usepackage[usenames, dvipsnames]{color}
\usepackage{color}
\usepackage{verbatim}
\usepackage{times}
\usepackage[T1]{fontenc}
\usepackage{aecompl}
\usepackage{url}
%\usepackage{aas_macros}
\usepackage{multirow}
\usepackage{tikz}
\usepackage{widetext}

\usepackage{hyperref}
\hypersetup{
    pdfnewwindow=true,      % links in new window
    colorlinks=true,       % false: boxed links; true: colored links
    linkcolor=blue,          % color of internal links
    citecolor=blue,        % color of links to bibliography
    filecolor=blue,      % color of file links
    urlcolor=blue           % color of external links
}

% AAS macros
\usepackage{aasmacro} % Ripped out of AAStex


% For script r
\usepackage{calligra}
\DeclareMathAlphabet{\mathcalligra}{T1}{calligra}{m}{n}
\DeclareFontShape{T1}{calligra}{m}{n}{<->s*[2.2]callig15}{}

% Bibliography:
\usepackage{natbib}
\bibliographystyle{apj}
%\bibliographystyle{mnras}

%%DANS ADS BIB SHORTHAND:
%%% Journal abbreviations.
\def\prd{PRD}
\def\apj{ApJ}                 % Astrophysical Journal
\def\apjl{ApJL}               % Astrophysical Journal, Letters
\def\apjs{ApJS}               % Astrophysical Journal, Supplement
\def\mnras{MNRAS}             % Monthly Notices of the RAS
\def\aap{A\&A}                % Astronomy and Astrophysics
\def\aaps{A\&AS}              % Astronomy and Astrophysics, Supplement 
\def\aj{AJ}                   % Astronomical Journal
\def\physrep{Phys.~Rep.}      % Physics Reports
\def\nat{Nature}              % Nature
\def\araa{ARA\&A}             % Annual Review of Astronomy and Astrophysics
\def\planss{planss}           % Planetary and Space Science
\def\ssr{SSR}                 % Space Science Reviews          
\def\sovast{Sov.~Astron.}     % Soviet Astronomy         
\def\canjphys{Can. J. Phys.}  % Canadian Journal of physics
\def\actaa{Acta Astron}  % Acta Astronomica  - polish astronomy and astrophysics
\def\icarus{Icarus}  % Solar System studies journal
\def\cqg{Class.~Quantum~Grav.}
\def\apss{Ap\&SS}                 % Astrophysics and Space Science 









%
% Definitions
%

\newcommand{\article}{\textsl{Thesis}}

% Stats / probability
\newcommand{\given}{\,|\,}
\newcommand{\norm}{\mathcal{N}}

% Maths
\newcommand{\dd}{\mathrm{d}}
\newcommand{\transpose}[1]{{#1}^{\mathsf{T}}}
\newcommand{\inverse}[1]{{#1}^{-1}}
\newcommand{\mean}[1]{\left< #1 \right>}
\newcommand{\ident}{\mathbb{1}} % identity matrix






%DANS DEFS
%lsim, gsim
\newcommand\lsim{\mathrel{\rlap{\lower4pt\hbox{\hskip1pt$\sim$}}
        \raise1pt\hbox{$<$}}}
\newcommand\gsim{\mathrel{\rlap{\lower4pt\hbox{\hskip1pt$\sim$}}
        \raise1pt\hbox{$>$}}} 
% Ion species with small caps.
\newcommand\ion[2]{#1{\thinspace\scshape#2}}% 
%ch2
%%APPENDIX DEFs FOR MN2E
\newcommand{\vr}{v_{\hat{r}}}
\newcommand{\vp}{v_{\hat{\phi}}}
\newcommand{\hr}{{\hat{r}}}
\newcommand{\hp}{{\hat{\phi}}}
%ch3
\newcommand{\scripty}[1]{\ensuremath{\mathcalligra{#1}}}
\def\sr{\scripty{r}}
\def\Mach{\mathcal{M}}
\def\bin{\rm{bin}}
\def\Mdot{\dot{M}}
%ch4
\def\obs{\rm{obs}}
\def\var{\rm{var}}
\def\QSO{\rm{QS)}}
%ch5
%L_Eddington
\def\Ledd{ L_{ \rm{Edd} } }
%alternate  Msun ~oh well
\def\Msun{ \rm M_{ \odot } }
\newcommand{\mnrasl}{MNRAS Lett.}
\renewcommand{\thefootnote}{\alph{footnote}}
\long\def\symbolfootnote[#1]#2{\begingroup%
\def\thefootnote{\fnsymbol{footnote}}\footnote[#1]{#2}\endgroup}
%ch6
%
%
%ch7
\def\nobf{}
\def\hb{\hat{\boldsymbol \beta}}
\def\bb{{\boldsymbol \beta}}
\def\bbd{{\boldsymbol  \beta^{\prime}}}
\def\bbR{{\boldsymbol \beta}_R}
\def\bR{\beta_R}
\def\gR{\gamma_R}
\def\g{\gamma}
\def\Res{\mathcal{R}}
\def\Sh{\mathcal{\tilde{S}}}
\def\Ch{\mathcal{\tilde{C}}}
\def\ChT{\mathcal{\tilde{C}}2}
\def\BM{{\bf B}_M}
\def\BR{{\bf B}_R}
\def\EM{{\bf E}_M}
\def\ER{{\bf E}_R}
\def\EE{{\bf E}}
\def\EB{{\bf B}}
\def\hr{\hat {\bf r}}
\def\rr{{\bf r}}
\def\hrc{\hat {\bf r}_c}
\def\rc{r_c}
\def\ba{{\bf a}}
\def\bV{{\bf V}}
\def\bm{{\bf m}}
\def\vb{v_{S, x}}
%ch 8
\def\g{\gamma}
\def\Res{\mathcal{R}}

\def\Evec{{\bf E}}
\def\Bvec{{\bf B}}
\def\Jvec{{\bf J}}
\def\hr{\hat {\bf r}}
\def\rr{{\bf r}}
\def\hrc{\hat {\bf r}_c}
\def\rc{r_c}
\def\bV{{\bf V}}
\def\bm{{\bf m}}
\def\bin{\rm{bin}}
\def\orb{\rm{orb}}
\def\Msun{{M_{\odot}}}
\newcommand{\BHcirc}[2][black,fill=black]{\tikz[baseline=-0.5ex]\draw[#1,radius=#2] (0,0) circle ;}%

\def\MBH{M}
\def\MNS{M_{\rm{NS}} }
\def\Lum{\mathcal{P}}

\def\rns{r}
\def\rbh{\scripty{r}}
\def\Rbin{r}
\def\RNS{R_{\rm NS}}

\def\bea{\begin{eqnarray}}
\def\eea{\end{eqnarray}}

\def\be{\begin{equation}}
\def\ee{\end{equation}}


% Milky Way shorthands
\newcommand{\mwdisk}{\textsl{Disk}}
\newcommand{\mwbulge}{\textsl{Bulge}}
\newcommand{\mwhalo}{\textsl{Halo}}

% Astronomy
\newcommand{\DM}{{\rm DM}}
\newcommand{\feh}{\ensuremath{{[{\rm Fe}/{\rm H}]}}}

% Unit shortcuts
\newcommand{\msun}{\ensuremath{\mathrm{M}_\odot}}
\newcommand{\kms}{\ensuremath{\mathrm{km}~\mathrm{s}^{-1}}}
\newcommand{\kpc}{\ensuremath{\mathrm{kpc}}}
\newcommand{\kmskpc}{\ensuremath{\mathrm{km}~\mathrm{s}^{-1}~\mathrm{kpc}^{-1}}}

% Misc.
\newcommand{\bs}[1]{\boldsymbol{#1}}
\newcommand{\todo}[1]{{\color{red} TODO: #1}}

% Packages / projects / programming
\newcommand{\package}[1]{\textsf{#1}}
\newcommand{\project}[1]{{\textsl{#1}}}
\newcommand{\rewinder}{\package{Rewinder}}
\newcommand{\streakline}{\package{Streakline}}
\newcommand{\superfreq}{\package{SuperFreq}}
\newcommand{\gaia}{\project{Gaia}}
\newcommand{\spitzer}{\project{Spitzer}}

\newcommand{\github}{\project{GitHub}}
\newcommand{\python}{\texttt{Python}}

%\usepackage{natbib}
%\usepackage{graphicx} % replaces epsfig
%% \usepackage{deluxetable52} % replaces deluxetable
%\usepackage{aasmacro} % replaces aastex_hack
%\usepackage{afterpage} % allows forced float output without breaking text flow
%\usepackage{pxfonts} % nice font
%\usepackage[nottoc,numbib]{tocbibind}
%\usepackage{setspace}
%%\usepackage{morefloats}
%\usepackage{float}
%\usepackage{verbatim}
%
%%\usepackage[margin=10pt,font={small,singlespacing},labelfont=bf,singlelinecheck=off]{caption}
%
%%added by Emily
%\usepackage{rotating}
%\usepackage{multirow}
%%\usepackage{mathabx}
%
%%added by Destry
%%\usepackage{rotating,amsmath}    % removed by Yuan --- it was causing errors
%\usepackage{lscape}
%\usepackage{verbatim}
%
%%added by Yuan
%\usepackage[3D]{movie15}
%\usepackage{hyperref}
%\usepackage{multirow}
%\usepackage[flushleft]{threeparttable}
%%\usepackage{epstopdf}
%
%% Added by Munier
%\usepackage{color}
%\usepackage{enumitem}
%\usepackage{booktabs}

\usepackage{setspace}
\doublespacing

\begin{document}
\title{Predicting Electromagnetic Signatures of Gravitational Wave Sources}
\author{Daniel J. D'Orazio}
\date{}
\maketitle


Scientific discovery is driven by observations. To date, the vast majority of
all such observations and corresponding scientific conclusions are founded on
the detection of photons, the messenger of the electromagnetic interaction.
This is changing as we enter the era of multi-messenger astronomy, where
observations of high energy neutrinos and cosmic rays will (and have already
begun to) teach us about the most energetic explosions in the universe and the
phenomenal events which trigger them, gamma-ray bursts, supernovae, and the
unknown \citep[\textit{e.g.}][]{Hirata:1987, Bionta:1987, ICECUBE:2013:detection}. 
Only months before completing this dissertation, on
September 14, 2015, the laser interferometer gravitational wave observatory
(LIGO) made the the first observation of gravitational waves, the messenger of
the gravitational interaction \citep{GW150914:2016}. The field of
gravitational wave astronomy will allow us to study the most extreme
gravitational environments in the universe, the collision of black holes, and
the fractions of a second following the Big Bang, otherwise inaccessible with
photons. The task of decoding the multi-lingual music of the electromagnetic
plus neutrino plus gravitational wave universe and making predictions for what
type of multi-messenger signals to expect and hence how to utilize multi-
messenger signals in order to maximize scientific returns now falls to the
burgeoning field of multi-messenger astronomy.

The work laid out in this dissertation is a contribution to this effort,
specifically to maximize our observations of gravitational wave sources by
predicting the nature of electromagnetic signatures that should accompany
them, or signify their existence beforehand. Such an endeavor not only
provides ways to find sources of gravitational waves and learn about their
operation, but also drives investigation into the astrophysics that creates
gravitational wave sources, and into the workings of physical processes in the
extreme environments that generate gravitational radiation.



This dissertation investigates the signatures of electromagnetic radiation
that may accompany two specific sources of gravitational radiation: the
inspiral and merger of massive black hole binaries (MBHBs) in galactic nuclei,
and the coalescence of black hole-neutron star pairs.



Part I considers the interaction of MBHBs, at sub-pc separations, with a
circumbinary gas disk. Accretion rates onto MBHBs are calculated from two-
dimensional hydrodynamical simulations as a function of the relative masses
of the black holes. The results are applied to interpretation of the recent,
sub-pc separation MBHB candidate in the nucleus of the periodically variable
quasar PG 1302-102. We advance an interpretation of the variability observed
in PG 1302-102 as being caused by Doppler boosted emission sourced by the
orbital velocity of the smaller black hole in a MBHB with disparate relative
masses.

Part II considers black hole-neutron star binaries in which the black hole is
large enough to swallow the neutron star whole before it is disrupted. As the
pair nears merger, orbital motion of the black hole through the magnetosphere
of the neutron star generates an electromotive force, a black-hole battery,
that could power luminosities large enough to make the merging pair observable
out to cosmic distances for magnetar-strength neutron star surface fields.
Fully analytic, relativistic solutions for vacuum fields of a magnetic dipole
near a horizon are given, and a mechanism for harnessing the power of the
black-hole battery is put forth in the form of a fireball emitting in hard
X-rays to $\gamma$-rays.


A more detailed synopsis of each chapter comprising this dissertation follows.
\\

\noindent
\textbf{Part I: Massive Black Hole Binaries}

\noindent
\textbf{Accretion into the central cavity of a circumbinary disk}

A near-equal-mass binary black hole can clear a central cavity in a
circumbinary accretion disk; however, previous works have revealed   accretion
streams entering this cavity.  Here we use 2D   hydrodynamical simulations to
study the accretion streams and their   periodic behavior.  In particular, we
perform a suite of   simulations, covering different binary mass ratios
$q=M_2/M_1$ in   the range $0.003 \leq q \leq 1$.  In each case, we follow the
system   for several thousand binary orbits, until it relaxes to a stable
accretion pattern.  We find the following results: (i) The   binary is
efficient in maintaining a low-density cavity. However, the time-averaged mass
accretion rate into the cavity, through narrow coherent accretion streams, is
suppressed by at most a factor of a few compared to a disk with a single black
hole with the same mass; (ii) for $q \gsim 0.05$, the accretion rate is
strongly modulated by the binary, and depending on the precise value   of $q$,
the power spectrum of the accretion rate shows either one,   two, or three
distinct periods; and (iii) for $q \lsim 0.05$, the   accretion rate becomes
steady, with no time-variations.  Most   binaries produced in galactic mergers
are expected to have $q\gsim   0.05$. If the luminosity of these binaries
tracks their accretion   rate, then a periodogram of their light-curve could
help in their   identification, and to constrain their mass ratio and disk
properties. \\


\noindent
\textbf{A transition in circumbinary accretion disks at a binary mass ratio of 1:25}

We study circumbinary accretion disks in the framework of the restricted
three-body problem (R3Bp) and via numerically solving the height-integrated
equations of viscous hydrodynamics. Varying the mass ratio of the binary, we
find a pronounced change in the behavior of the disk near mass ratio $q \equiv
M_s/M_p \sim 0.04$.  For mass ratios above $q=0.04$, solutions for the
hydrodynamic flow transition from steady, to strongly-fluctuating; a narrow
annular gap in the surface density around the secondary's orbit changes to a
hollow central cavity; and a spatial symmetry is lost, resulting in a lopsided
disk. This phase transition is coincident with the mass ratio above which
stable orbits do not exist around the L4 and L5 equilibrium points of the R3B
problem.  Using the DISCO code, we find that for thin disks, for which a gap
or cavity can remain open, the mass ratio of the transition is relatively
insensitive to disk viscosity and pressure.  The $q=0.04$ transition has
relevance for the evolution of massive black hole binary+disk systems at the
centers of galactic nuclei, as well as for young stellar binaries and possibly
planets around brown dwarfs.\\


\noindent
\textbf{A reduced orbital period for the massive black hole binary candidate in the quasar PG 1302-102?}

\cite{Graham+2015a} have detected a 5.2 yr periodic optical variability of the
quasar PG~1302-102 at redshift $z=0.3$, which they interpret as the redshifted
orbital period $(1+z)t_{\rm bin}$ of a putative massive black hole binary
(MBHB). Here, we consider the implications of a $3-8$ times shorter orbital
period, suggested by hydrodynamical simulations of circumbinary disks with
nearly equal--mass MBHBs ($q\equiv M_2/M_1\gsim 0.3$).  With the corresponding
$2-4$ times tighter binary separation, PG~1302 would be undergoing
gravitational wave dominated inspiral, and serve as a proof that the black
holes can be fueled and produce bright emission even in this late stage of the
merger. The expected fraction of binaries with the shorter $t_{\rm bin}$,
among bright quasars, would be reduced by one to two orders of magnitude,
compared to the 5.2 yr period, in better agreement with the rarity of
candidates reported by \cite{Graham+2015a}.  Finally, shorter periods would
imply higher binary speeds, possibly imprinting periodicity on the light
curves from relativistic beaming, as well as measurable relativistic effects
on the Fe K $\alpha$ line.  The circumbinary disk model predicts additional
periodic variability on time-scales of $t_{\rm bin}$ and $\approx 0.5 t_{\rm
bin}$, as well as periodic variation of broad line widths and offsets relative
to the narrow lines, which are consistent with the observations.  Future
observations will be able to test these predictions and hence the circumbinary
disk hypothesis for PG~1302.\\

\noindent
\textbf{Relativistic boost as the cause of the periodicity in a massive
black hole binary candidate} 

As most large galaxies contain a central black hole, and as galaxies   often
merge \cite{KormendyHo2013}, black hole binaries are expected   to be common
in galactic nuclei \cite{Begel:Blan:Rees:1980}.  Although they cannot   be
imaged, periodicities in the light curves of quasars have been interpreted as
evidence for binaries \cite{Komossa:Rev06,Valtonen+2008,Liu:7RsMBHB:2015},
most recently in PG~1302-102, with a short rest-frame optical period of 4~yr
\citep{Graham+2015a}. If the orbital period matches this value,   then for the
range of estimated black hole masses the components   would be separated by
0.007-0.017 pc, implying relativistic orbital   speeds. There has been much
debate over whether black hole orbits   could be smaller than 1 pc
\cite{Milosavljevic:2003:FPcP}. Here we show that the   amplitude and the
sinusoid-like shape of the variability of PG~1302-102 can be fit by
relativistic Doppler boosting of emission   from a compact, steadily
accreting, unequal-mass binary.  We predict   that brightness variations in
the ultraviolet light curve track those in the optical, but with a 2-3 times
larger amplitude. This   prediction is relatively insensitive to the details
of the emission   process, and is consistent with archival UV data.  Follow-up
UV and   optical observations in the next few years can test this prediction
and confirm the existence of a binary black hole in the relativistic   regime.\\


\noindent
\textbf{Reverberation of doppler boosted emission from massive black hole
binaries: A lighthouse in the dust} 

We consider the reverberation of AGN emission by dust in the context of
massive black hole binaries which emit periodic continuum emission due to both
spatially isotropic variations and anisotropic variations caused by orbital
relativistic Doppler boosting. We develop the first models for IR emission
from AGN harboring such MBHBs, providing an additional test to vet the Doppler
boosting model for MBHB candidates, and a tool for constraining properties of
the dusty environments around MBHBs. We show that the phase, amplitude, and
average brightness of reverberated IR radiation is dependent on the ratio of
light travel time across the emitting dust region as well as dust geometric
properties, and in the case of Doppler boosted emission, the relative
inclination of binary orbital plane and dust torus. We determine also that, in
the Doppler boost model, IR variability and UV/optical variability need not be
coincident; UV emission could be steady, while the IR is periodically
modulated, or vice versa. IR surveys should look for such orphan IR
variability.

In the near future these model will help to corroborate evidence for the
growing number of (presently $\gsim 100$) MBHB candidates \citep{Graham+2015b,
Charisi+2016}, find new candidates, and also constrain their physical
properties and the properties of their surrounding, dusty environments. \\

\clearpage
\noindent
\textbf{Part II: Black Hole-Neutron Star Binaries}

\noindent
\textbf{Big black hole, little neutron star: magnetic dipole fields in the Rindler spacetime}

As a black hole and neutron star approach during inspiral, the field lines of
a magnetized neutron star eventually thread the black hole event horizon and a
short-lived electromagnetic circuit is established. The black hole acts as a
battery that provides power to the circuit, thereby lighting up the pair just
before merger.  Although originally suggested as an electromagnetic
counterpart to gravitational-wave detection, a black hole battery is of more
general interest as a novel luminous astrophysical source. To aid in the
theoretical understanding, we present analytic solutions for the
electromagnetic fields of a magnetic dipole in the presence of an event
horizon. In the limit that the neutron star is very close to a Schwarzschild
horizon, the Rindler limit, we can solve Maxwell's equations exactly for a
magnetic dipole on an arbitrary worldline. We present these solutions here and
investigate a proxy for a small segment of the neutron star orbit around a big
black hole. We find that the voltage the black hole battery can provide is in
the range $\sim 10^{16}$ statvolts with a projected luminosity of $10^{42}$
ergs/s for an $M=10M_\odot$ black hole, a neutron star with a B-field of
$10^{12} G$, and an orbital velocity $\sim 0.5 c$ at a distance of $3M$ from
the horizon. Larger black holes provide less power for binary separations at a
fixed number of gravitational radii.  The black hole-neutron star system
therefore has a significant power supply to light up various elements in the
circuit possibly powering bursts, jets, beamed radiation, or even a hot spot
on the neutron star crust.\\

\noindent
\textbf{Bright transients from strongly magnetized neutron star - black hole mergers}

Direct detection of black hole-neutron star pairs is anticipated with the
advent of aLIGO.  Electromagnetic counterparts may be crucial for a confident
gravitational-wave detection as well as for extraction of astronomical
information. Yet black hole-neutron star star pairs are notoriously dark and
so inaccessible to telescopes.  Contrary to this expectation, a bright
electromagnetic transient, introduced in the previous chapter, can occur in
the final moments before merger as long as the neutron star is highly
magnetized.  The orbital motion of the neutron star magnet creates a Faraday
flux and corresponding power available for luminosity. A spectrum of curvature
radiation ramps up until the rapid injection of energy ignites a fireball,
which would appear as an energetic blackbody peaking in the X-ray to $\gamma$
- rays for neutron star field strengths ranging from $10^{12}$G to $10^{16}$G
respectively and a $10M_{\odot} $ black hole. The fireball event may last from
a few milliseconds to a few seconds depending on the neutron star magnetic
field strength, and may be observable with the Fermi Gamma-Ray Burst Monitor
with a rate up to a few per year for neutron star field strengths $\gtrsim
10^{14}$G.  We also discuss a possible decaying post-merger event which could
accompany this signal.  As an electromagnetic counterpart to these otherwise
dark pairs, the black-hole battery should be of great value to the development
of multi- messenger astronomy in the era of aLIGO.



\markboth{Bibliography}{Bibliography}
\bibliography{refs/refs}
\end{document}