\thispagestyle{empty}
\begin{center}

{\Large \bf ABSTRACT}

\vspace{.35in}
{\large \bf \thesistitle}

\vspace{.35in}

{\large Daniel John D'Orazio} \\
\vspace{.35in}
\end{center}
%

This dissertation investigates the signatures of electromagnetic radiation
that may accompany two specific sources of gravitational radiation: the
inspiral and merger of massive black hole binaries (MBHBs) in galactic
nuclei, and the coalescence of neutron star black hole (NSBH) pairs. 

Part I considers the interaction of MBHBs, at sub-pc separations, with a
circumbinary gas disk. Accretion rates onto the MBHB are calculated from two-dimensional 
hydrodynamical simulations as a function of the relative masses of
the black holes. The results are applied to interpretation of the recent, sub-pc 
separation MBHB candidate in the nucleus of the periodically variable
Quasar PG 1302-102. We advance an interpretation of the variability observed
in PG 1302-102 as being caused by Doppler boosted emission sourced by the
orbital velocity of the smaller black hole in a MBHB with disparate relative masses.

Part II considers NSBH binaries in which the black hole is large enough
swallow the NS whole before it is disrupted. As the pair nears merger, orbital
motion of the black hole through the magnetosphere of the NS generates an
electromotive force, a BH-battery, that could power luminosities large enough
to make the merging pair observable out to cosmic distances for magnetar-
strength neutron star surface fields. Relativistic solutions for vacuum fields
of a magnetic dipole near a horizon are given, and a mechanism for harnessing
the power of the black-hole battery is put forth in the form of a fireball emitting in 
hard X-rays to $\gamma$-rays.
