\thispagestyle{empty}
\begin{center}

{\Large \bf ABSTRACT}

\vspace{.35in}
{\large \bf \thesistitle}

\vspace{.35in}

{\large Daniel John D'Orazio} \\
\vspace{.35in}
\end{center}
%
The first direct-detection of gravitational waves emanating from the merger of two $30 \msun$ black holes 

We develop theoretical predictions for the electromagnetic signatures of two promising sources of gravitational wave radiation. Part I of this thesis concerns the inspiral of massive black hole binaries (MBHBs) in the centers of galactic nuclei. Part II treats the merger of a neutron star (NS) and a stellar black hole (BH) in the case where the magnetized NS is swallowed whole by the BH.

The mergers of MBHBs, both individual events as well as a stochastic background, will be detectable in gravitational radiation by future space-based interferometers and by the Pulsar timing arrays. Because the merger of MBH harboring galaxies can bring gas and both BHs to the nucleus of the newly formed galaxy, we study the interaction of gas and MBHBs and the bright EM radiation that could be liberated in such systems via accretion. We study accretion onto MBHBs using tools ranging from analytic treatments of steady disks up to multi-dimensional hydrodynamical simulations of the binary and disk system. The work in this thesis characterizes the accretion magnitude and variability onto the binary as a function of the ratio of BH masses which make up the binary (the binary mass ratio). This thesis makes the first predictions for the mass ratio dependent spectrum of accretion variability and makes the first application of these predictions to interpret the first MBHB candidates discovered in EM time domain surveys. The final chapter of this Part I of this thesis begins a journey into signatures of binary accretion that could be present in the dusty media surrounding the binary.

The merger of a neutron star and black hole is detectable by the Laser Interferometer Gravitational Wave Observatory (LIGO).  